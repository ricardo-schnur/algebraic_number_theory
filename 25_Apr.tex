\rhead{25 April 2018}


\begin{proof}
$p^r_{m,n} \colon \O / \p^n \rightarrow \O/ \p^m$ sei die natürliche Projektion für $m\leq n$
und 
\[ h_n \colon \mathcal{L} = \limproj_{n\in\N} \O/\p^n \to \O/\p^n
\]
die Projektionsabbildung des Limes.
\begin{itemize} 
\item[(1)] Definiere eine Abbildung $\Phi \colon \hat{\O} \to\mathcal{L}$:
	
		Verwende den Isomorphismus $\alpha_n \colon \hat{\O}/ \hat{\p}^n \to \O / \p$ aus Proposition 5.5.2 und definiere 
		\[ \Phi_n \colon \hat{\O} \to \O / \p^n, \, x \mapsto \alpha_n \left(  x \mod \hat{\p}^n \right).
		\]
		Dann ist $\Phi_n$ ein Ringhomomorphismus und die $\{ \Phi_n \}$ passen zusammen mit den Projektionen $p_{m,n}$.
		
		\bigskip Zeige noch: $\Phi_n$ ist stetig
		
		Sei $U \subset \O/\p^n$ und $V=\Phi_n^{-1}(U)$, Es gilt $\ker \Phi_n = \hat{p}^n$. Für $x\in V$ und $\varepsilon_n = q^{-n}$ ist dann
		\begin{align*}
			K_{\varepsilon_n}(x) 
			&= \{ y \in \hat{\O} \,|\, \abs{y-x}< \varepsilon_n  \} \\
			&= \{ y \in \hat{\O} \,|\, \hat{v}(y-x) > n  \} \\
			&= x+ \hat{\p}^{n+1}.
		\end{align*}
		Folglich gilt $\Phi(n)( K_{\varepsilon_n}(x)  ) = \{x\}$ und daher $K_{\varepsilon_n}(x)  \subset V$. Also ist $V$ offen.
		Nach der Definition von projektivem Limes induziert $\{\Phi_n \}$ also einen stetigen Ringhomomorphismus  $\Phi \colon \hat{\O} \to\mathcal{L}$.
\item[(2)] $\Phi$ ist injektiv:
\begin{align*}
\Phi(x) = 0
&\Leftrightarrow \Phi_n(x) = 0 \text{ für alle } n \in \N \\
&\Leftrightarrow x \mod \hat{\p}^n  \text{ für alle } n \in \N \\
&\Leftrightarrow x=0
\end{align*} 
\item[(3)] $\Phi$ ist surjektiv:

Sei $s=(\overline{s_n}) \in \mathcal{L}$, also $p^r_{n,m}(\overline{s_m}) = \overline{s_n}$ für $n\leq m$. Wähle für jedes $\overline{s_n}$ ein Urbild $s_n \in \O$. Es gilt also $s_m-s_n \in \p^n$ für $m\geq n$. Also ist $(s_n)$ eine Cauchy-Folge und es existiert $x=\lim_n s_n \in \hat{\O}$.
Dann gilt $\Phi(x)=s$, denn:

Es gilt für alle $m\geq n$, dass $s_n \equiv s_m \mod \p^n$. Wegen $s_n\to x$ existiert $k_0$ mit $v(x-s_n) \geq n$. Wähle $k\geq k_0,n$, dann gilt
\[ x \equiv s_k \equiv s_n \mod \hat{\p}^n
\]
und damit $\Phi_n(x)= \overline{s_n}$ so, dass $\Phi(x)=s$.
\item[(3)] $\Phi$ ist offen, d.h. für alle $U\subset \hat{\O}$ offen ist $\Phi(U)$ offen:

Da $(\hat{\O},+)$ eine topologische Gruppe ist, genügt es, dies für alle $U$ einer Umgebungsbasis von $\O$ zu überprüfen und
\[ \B = \{ K_{q^{-1}}(0) \,|\, n \in \N_0\}
= \{ \hat{\p}^n \,|\, n \in \N \}
\]
ist eine Umgebungsbasis von $0$. Für $\hat{\p}^n \in \B$ gilt:
\begin{align*}
\Phi(\hat{\p}^n)
&= \{ (\overline{s_k}) \in \mathcal{L} \,|\, \overline{s_k}=0 \text{ für } k\leq n  \} \\
&= h_n^{-1} ( \{0\}) \text{ offen in } \mathcal{L}
\end{align*}
Somit ist $\Phi$ offen.
\end{itemize}
\end{proof}


\begin{Bem}
Die Mengen
\[ U_n = \{0\} \times \cdots \times \{0\} \times \O / \p^{n+1} \times \cdots
\]
als offene Teilmengen von $\mathcal{L} = \limproj_{n\in\N} \O/\p^n$ bilden eine Umgebungsbasis der $0$.
\end{Bem}


\begin{Bem}
Der Homomorphismus 
\[ \O^\times \to \left( \O / \p^n \right)^\times, \, u \mapsto u \mod \p^n 
\]
hat als Kern $U^{(n)} = 1+\p^n$ und ist surjektiv. Somit ist $U^{(n)}$ eine Untergruppe von $\O^\times$ und
\[ \O^\times / U^{(n)} \cong \left( \O / \p^n \right)^\times.
\]
\end{Bem}

\begin{Kor}
Als topologische Gruppen gilt
\[ \hat{O}^\times = \limproj_{n\in\N} \O^\times / U^{(n)} .
\]
\end{Kor}

\begin{proof}
Der Isomorphismus $\Phi \colon \hat{\O} \limproj_{n\in\N} \O/\p^n$ schränkt sich ein zu einem Isomorphismus $\hat{O}^\times \to \left(  \limproj_{n\in\N} \O/\p^n\right)^\times$. Damit folgt
\begin{align*}
\left(  \limproj_{n\in\N} \O/\p^n\right)^\times
&\cong  \limproj_{n\in\N} \left(  \O/\p^n\right)^\times
\cong \limproj_{n\in\N} \O^\times / U^{(n)} .
\end{align*}
\end{proof}

\begin{Kor}
$\Z_p \cong \limproj_{n\in\N}  \Z/p^n\Z$
\end{Kor}


\textbf{Ziel:} Wähle zu jedem $x\in\Z_p$ eine eindeutige Folge $(s_n)$ in $\Z$ mit $s_n\to x$ und zwar so, dass $v(s_{n+1}-s_n)> n$, d.h.
\[ s_{n+1} = s_n + a_{n} p^n. 
\]
Damit hat $s_{n+1}$ die Form
\[ s_{n+1} = a_0 +a_1p + a_2p^2 + \cdots a_np^n.
\]

\bigskip\textbf{Idee:} Wenn wir $a_i \in \{ 0,1, \dots, p-1 \}$ fordern, dann geht das auf eindeutige Weise.


\begin{Prop}
Sei $R\subset\O$ ein Repräsentantensystem für $\kappa = \O / \p$ mit $0\in R$.
\begin{enumerate}[(i)]
\item Für alle $x\in\hat{\O}$ existiert eine eindeutige Folge $(a_n)_{n\in\N_0}$ in $R$ mit
\[ x \equiv s_{n} = a_0 +a_1\Pi + a_2\Pi^2 + \cdots a_{n-1}\Pi^{n-1} \mod \hat{\p}^n
\]
für alle $n\in\N$ und es gilt $s_n \to x$ bezüglich $\abs{\cdot}$.
\item Für $x\in\hat{K}^\times$ existiert eine analoge Folge mit
\[  s_{n} = a_m\Pi^m +a_{m+1}\Pi^{m+1} +  \cdots a_{n}\Pi^{n} 
\]
für $n\geq m$ so, dass $s_n \equiv x \mod \hat{p}^n$ und $s_n \to x$. Hierbei ist $m=v(x)$.
\end{enumerate}
\end{Prop}

\begin{proof}
\enquote{(i)} Nach Proposition 5.5.2 existiert ein eindeutiges $a_0 \in R$ mit $a_0\equiv x \mod \hat{p}^n$. Definiere $a_n$ rekursiv wie folgt:

Es gilt bereits $x \equiv s_{n} = a_0 +a_1\Pi + a_2\Pi^2 + \cdots a_{n-1}\Pi^{n-1} \mod \hat{\p}^n$. Dann ist
\[ x-s_n = u_n \Pi^n
\]
für ein $u_n \in \hat{\O}$. Nach Proposition 5.5.2 lässt sich $u_n$ schreiben als
\[ a_n + r\Pi
\]
mit eindeutigem $a_n\in R$ und einem $r \in \hat{\O}$. Also
\begin{align*}
x  
&= a_0 +a_1\Pi + a_2\Pi^2 + \cdots a_{n-1}\Pi^{n-1}+ u_n \Pi^n \\
&= a_0 +a_1\Pi + a_2\Pi^2 + \cdots a_{n-1}\Pi^{n-1}  + a_n \Pi^n +r \Pi^{n+1}.
\end{align*}
Also leistet $a_n$ das Gewünschte, d.h.
\[ x \equiv a_0 +a_1\Pi + a_2\Pi^2 + \cdots a_{n}\Pi^{n} = s_{n+1} \mod \hat{\p}^{n+1}.
\]Schließlich $v(x-s_n) \geq n+1$ und damit $s_n \to x$.

\bigskip\enquote{(ii)} Sei $x\in\hat{K}$. Dann ist $x=u\Pi^m$ mit einem $u\in\O^\times$ und $m=v(x)$.
Erhalte aus (i) eine Folge $(a_n')$ für $u$ mit
\[ a_0' + a_1'\Pi + \cdots a_n'\Pi^n \to u.
\]
Wegen $u\in\O^\times$ ist $a_0' \neq 0$. Erhalte also
\[ \Pi^m \left(a_0' + a_1'\Pi + \cdots a_n'\Pi^n  \right) \to x
\]
und setzte $a_n = a_{n+m}'$. Dann leistet $(a_n)$ das Gewünschte.
\end{proof}


\begin{Bsp}
Sei $K=\Q$ mit $p$-adischer Norm $\abs{\cdot}_p$.
\[\begin{array}{ccccc}
\hat{K} & & \limproj \Z /p^n \Z & & \{0, \dots, p-1  \}^{N_0} \\
-1 & \leftrightarrow & (p-1, p-1, p-1, \dots) & \leftrightarrow & p-1 + (p-1)p +(p-1)p^2 + \cdots
\end{array}
\]
\end{Bsp}

\begin{Kor}
Es gilt
\begin{align*}
\Z_p 
&= \{ a_0 + a_1p +a_2p^2 + \cdots \,|\, a_i \in \{0,\dots,p-1\}  \} \\
&=\{0, \dots, p-1  \}^{N_0}
\end{align*}
und
\begin{align*}
\Q_p 
&= \{ a_mp^m + a_{m-1}p^{m-1}+ a_{m-2}p^{m-2}  + \cdots \,|\, m \in \Z,  a_i \in \{0,\dots,p-1\}, a_m \neq 0  \} \\
&=\Quot(\Z_p).
\end{align*}
Addition und Multiplikation ergibt sich wie bei Potenzreihen beziehungsweise Laurentreihen.
\end{Kor}


\begin{Bsp}
Seien $K=k(T) = \Quot+(k[T])$ und
\[ v \colon K \to \Z\cup \{\infty \}, \, \frac{p}{q} \mapsto \deg(p) - \deg(q).
\]
Weiter sei $\O = k[T] \supset \p = (T)$ und $\Pi = T$. Dann ist
$\hat{\O} = k[[T]]$ der Ring der Potenzreihen und $\hat{K} = k((T))$ der Körper der Laurentreihen.
\end{Bsp}



