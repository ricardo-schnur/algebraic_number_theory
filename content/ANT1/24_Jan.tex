
\rhead{24 January 2018}
\begin{proof}
\begin{enumerate}[i)]
\item \checkmark
\item \checkmark
\item Consider affine variety $V(J)\supseteq V \Rightarrow J \subseteq I(V) \Rightarrow V(J)\supseteq V(I(V))$
\item \glqq $\Rightarrow$\grqq: Suppose $f \in I(V_2) \Rightarrow \forall x \in V_1 \subseteq V_2: f(x)=0 \Rightarrow f \in I(V_2)$.\\
\glqq $\Leftarrow$ \grqq: $I_2 \stackrel{ii)}{\subseteq} I(V(I_2))=I(V_2) \subseteq I(V_1)$ Hence: $x \in V_1, f \in I_2 \Rightarrow f(x)=0$. Thus $x \in V_2 =V(I_2)$.
\item \checkmark
\end{enumerate}
\end{proof}

\underline{\textbf{Hilberts Nullstellensatz}} (without proof)\\
$k$ is algebraically closed, $I$ ideal in $k[X_1, \dots, X_n]$.\\
Then $I(V(I))=\sqrt{I}:=\{f \in k[X_1, \dots, X_n] \ | \ \exists l \in \N \text{ with } f^e \in I\}$.\\
\vspace*{\baselineskip}

\underline{Idea:} Define a topology on $\Aff^n(k)$.

\begin{Bemdef}
The affine varieties define the closed sets of a topology called \emph{\underline{Zariski topology}}.
\end{Bemdef}

\begin{proof}
\begin{itemize}
\item $\Aff^n(k)=V((0))$ and $\emptyset=V((1))$.
\item $V_1=V(I_1)$ and $V_2=V(I_2)$ affine varieties $\Rightarrow V_1 \cup V_2 \stackrel{(!)}{=} V(I_1 \cdot I_2)\stackrel{(!)}{=}V(I_1 \cap I_2)$
\item $V_j=V(I_j)$ family of affine varieties with $j \in J \Rightarrow \cap_{j \in J} V_j \stackrel{(!)}{=} V(\sum_{j\in J} I_j)$.
\end{itemize}
\end{proof}

\begin{Bsp}
In $\Aff^1(k)$ a subset is closed $\iff$ it is finite or $\emptyset$ or $\Aff^1(k)$.
\end{Bsp}

\begin{defi}
A topological space $X$ is called \emph{\underline{irreducible}} $:\iff X=A\cup B$ with $A,B$ closed implies $X=A$ or $X=B$.\\
Otherwise $X$ is called \emph{\underline{reducible}}
\end{defi}

\begin{Bsp}
$V=V(X\cdot Y)=V(X) \cup V(Y)$ is reducible. \begin{tikzpicture}
\draw (-1,0) -- (1,0) node[anchor=west]{$V(Y)$};
\draw (0,-1) -- (0,1) node[anchor=south]{$V(X)$};
\end{tikzpicture}
\end{Bsp}

\begin{Prop}
An affine variety $V$ is irreducible $\iff I(V)$ is a prime ideal.
\end{Prop}

\begin{proof}
\glqq $\Rightarrow$\grqq: Consider $f,g \in k[X_1, \dots, X_n]$ with $f,g \in I(V)$.\\
Suppose: $f \not \in I(V)$. Show that $g \in I(V)$.\\
$f \not \in I(V) \Rightarrow \exists f(x) \neq 0 \Rightarrow V \not\subseteq V(f) (\star)$\\
Observe: $V \subseteq V(f\cdot g) =V(f) \cup V(g) \Rightarrow V=\underbrace{(V(f) \cap V)}_{\text{closed}} \cup \underbrace{(V(g) \cap V)}_{\text{closed}}$\\
$\stackrel{(\star)}{\Rightarrow} V(g) \cap V =V \Rightarrow V \subseteq V(g) \Rightarrow g \in I(V)$.\\
\glqq $\Leftarrow$\grqq: Suppose $V=V_1 \cup V_2$ with $V_1 =V(I_1)$ and $V_2=V(I_2)$.\\
Hence: $V=V(I_1) \cup V(I_2)=V(I_1 I_2)$ and thus $I_1 \cdot I_2 \stackrel{2.7}{\subseteq} I(V(I_1 \cdot I_2)) = I(V)$.\\
Suppose: $V_1 \neq V \Rightarrow \exists x \in V: f \in I_1$ with $f(x)\neq 0$.\\
Hence $f \not \in I(V) \forall g \in I_2: f\cdot g \in I_1 I_2 \subseteq I(V) \stackrel{I(V) \text{ prime}}{\Longrightarrow} g \in I(V)$.\\
Hence $I_2 \subseteq I(V) \Rightarrow V_2=V(I_2)\supseteq V(I(V)) \supseteq V \Rightarrow V_2=V$.
\end{proof}

\begin{Bem}
An affine variety $V$ is irreducible $\iff k[V]=k[X_1, \dots, X_n]/I(V)$ is an integral domain.
\end{Bem}

From now on $k$ is always algebraically closed and all affine varieties are irreducible.

\begin{Bsp}
$V=\Aff^1(k), I(V)=(0), k[V]=k[X]$
\end{Bsp}

\begin{defi}
Let $U\subseteq V$ be an open subset of $V$. Define
\begin{align*}
\O(U):=\{\varphi: U \to k \ | \ &\forall z \in U \  \exists \text{ open neighbourhood } U_z \ni z \stackrel{\text{open}}{\subseteq} V,\ \exists f,g, \in k[V]\\
& \text{ with } \forall x \in U_z: g(x) \neq 0 \text{ and } \varphi(x)=\frac{f(x)}{g(x)}\}
\end{align*}
as the \emph{\underline{set of regular functions on U}}.
\end{defi}

\begin{Bem*}
$\O$ defines a sheaf on $V$.
\end{Bem*}

\begin{defi}
For $z \in V$ we define the \emph{\underline{local ring}} $\O_z$ as follows:
\[\O_z:=\{(U,f) \ | \ U \text{ open neighbourhood of } z, f \in \O(U)\}/\sim\]
with $(U_1,f_1) \sim (U_2, f_2) \iff f_1|_{U_1 \cap U_2} = f_2|_{U_1 \cap U_2}$.
\end{defi}


\begin{Bem*}
\begin{itemize}
\item $\O_z$ is the stalk of the sheaf $\O$ in $z$.
\item $\O_z = k[V]_{m_z}$ here $m_z=\{f \in k[V] \ | \ f(z)=0\}$
\end{itemize}
\end{Bem*}

\begin{Bem}
$m_z$ as defined above is a maximal ideal in $k[V]$.
\end{Bem}

\begin{proof}
$\varphi_u: k[V] = k[X_1, \dots, X_n] / I(V) \twoheadrightarrow k \ , \  f\mapsto f(z)$\\
is a $k$-algebra homomorphism, which is surjective.\\
Hence $m_z=\ker(\varphi_z)$ is a maximal ideal.
\end{proof}

\begin{Bem}
In particular $\O_z$ is a local ring.
\end{Bem}

\begin{defi}
The field of \emph{\underline{rational functions}} $\Rat(V)$:
\[\Rat(V):=\{(U,f) \ | \ U \text{ open in } V, f\in \O(U) \}/ \sim \]
with $(U_1, f_1) \sim (U_2, f_2) \iff f_1|_{U_1 \cap U_2} = f_2|_{U_1 \cap U_2}$.
\end{defi}

\begin{Bem*}[without proof]
$\Rat(V) = \Quot(k[V])$
\end{Bem*}

\begin{conc}
Still assume $k$ is algebraically closed, $V$ affine variety.\\
Then we have the following correspondences:
\begin{align*}
\text{closed subsets of } V &\stackrel{1:1}{\leftrightarrow} \text{ radical ideals in } k[V]\\
V' &\mapsto I(V')\\
\\
\text{ irreducible closed subsets of } V &\stackrel{1:1}{\leftrightarrow} \text{prime ideals}\\
\text{points} &\stackrel{1:1}{\leftrightarrow} \text{ maximal ideals in } k[V]\\
x &\mapsto I(\{x\})=m_x
\end{align*}

Furthermore we have:
\begin{itemize}
\item $V_1, V_2$ closed subsets of $V$: $V_1 \subseteq V_2 \iff I(V_1) \supseteq I(V_2)$\\
In particular: \hspace*{2cm} $x \in V_1 \iff m_x \supseteq I(V_1)$.
\item $\a$ ideal in $k[V]:$\\
$V(\a)=\{z \in V \ | \ \forall f \in \a: f(z)=0\}$\\
$\hphantom{V(\a)}=\{z \in V \ | \ \forall f \in \a: f \in m_z \} =\{z \in V \ | \ \a \subseteq m_z\}$
\item $S\subseteq V: I(S) = \{f \in k[V] \ | \ \forall s \in S: f(s)=0\}=\bigcap_{s \in S} m_s.$
\end{itemize}
\end{conc}