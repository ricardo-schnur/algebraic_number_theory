\rhead{31 January 2018}
\begin{Bem}
$R$ now an arbitrary ring.\\
The elements of $R$ define functions on $\Spec(R)$ as follows:
\[f: p \mapsto f \mod p \in R_p/p=:\underbrace{\kappa(p)}_{\emph{\underline{\text{residue field of } p}}}\]
\end{Bem}

\begin{Bsp}
$R=\Z, X=\Spec(\Z)=\{(p) \ | \ p \text{ prime } \} \cup \{(0)\}$. Observe:
\begin{itemize}
\item The $(p)$ are closed points.
\item $\bar{\{(0=\}} = \Spec(\Z)$
\end{itemize}
\end{Bsp}


\begin{defi}
Let $U \subseteq \Spec(R)$ be an open subset. Define
\begin{align*}
\O(U):=\{s=(s_p)_{s \in U} \in \prod_{p \in U} R_p \ | \ & \forall p \ \exists U_p \text{ open nbhd. of } p \text{ and } f,g \in R \text{ with}:\\
&\forall q \in U_p: \underbrace{g(q)}_{=g \mod q} \underbrace{\neq 0}_{\iff g \not \in q} \text{ and } s_q=\frac{f}{g} \text{ in } R_q\}
\end{align*}
Observe: $\O$ is a sheaf.
\end{defi}

\begin{defi}
The pair $(\Spec R, \O)$ is called an \emph{\underline{affine scheme}}.\\
It is called \emph{\underline{noetherian}} $\iff R$ is noetherian.\\
Its \emph{\underline{dimension}} is the dimension of $R$.
\end{defi}

\begin{Bsp}
\begin{enumerate}[i)]
\item $R=k[V]$ for some affine variety $V$ with $k$ is algebraically closed\\
$\Rightarrow V \hookrightarrow \Spec(R)$ as set of closed points.
\item $R=K$ field $\Rightarrow \Spec(K)=\{(0)\}: \O(\{0\})=K$\\
$\underline{/!\backslash} \Spec(K)$ as affine scheme depends on $K$.
\item $R$ discrete valuation ring with maximal ideal $m \Rightarrow \Spec(R)=\{(0),m\}$\\
Closed subsets: $\emptyset, \{m\}, \Spec(R)$\\
Open subsets: $\Spec(R), \{0\}, \emptyset$\\
$\O(\{0\})=\Quot(R), \O(\Spec(R))=R_m=R$ \todo{Bilder?}
\end{enumerate}
\end{Bsp}

\begin{defi}
For a one-dimensional (!) noetherian domain $R$ we say that the associated affine scheme ($\Spec R \O$) is \emph{\underline{regular}} $:\iff$ All local rings $R_p (p $ prime in $R, p \neq (0)$) are discrete valuation rings.
\end{defi}

\begin{Beob}
For a domain $R$ we have: $(\Spec R, \O)$ is a noetherian, one-dimensional, regular affine scheme $\iff R$ is a Dedekind domain.
\end{Beob}

\section{A \glqq general\grqq example}
\begin{Bsp}
Consider $V_1:=V(f_\lambda) \subseteq \Aff^2(\C)$ with $f_\lambda=Y^2-X(X-1)(X-\lambda)$ and $\lambda \in \C \setminus \{0,1\}, V_2:\Aff^1(\C), f:V_1 \to V_2 \ , \ (x,y) \mapsto x$
\[ \begin{tikzpicture}[scale=0.5]
\draw[brown](-1.5,0) -- (2,0);
\draw(0,-1.5) -- (0,2.5);
\draw[smooth, domain=0:1, variable=\x,blue] plot ({\x},{sqrt(\x*(\x-1)*(\x-2))});
\draw[smooth, domain=2:2.5, variable=\x,blue] plot ({\x},{sqrt(\x*(\x-1)*(\x-2))});
\draw[smooth, domain=0:1, variable=\x,blue] plot ({\x},{-sqrt(\x*(\x-1)*(\x-2))});
\draw[smooth, domain=2:2.5, variable=\x,blue] plot ({\x},{-sqrt(\x*(\x-1)*(\x-2))});
\end{tikzpicture}
\]
\end{Bsp}

\begin{Bem}
\begin{enumerate}[i)]
\item $x \in \Aff^1(\C)$ has two preimages $\iff x \not \in \{0,1,\lambda\}$
\item $x \in \Aff^1(\C)$ has one preimage $\iff x \in \{0,1,\lambda\}$, namely $(x,0)$.
\end{enumerate}
\end{Bem}

\begin{Bem}
Recall: $\O_1:=\C[V_1]=\C[X,Y]/(f_\lambda), \O_2:=\C[V_2]=\C[X]$\\
The map $f$ induces:
\begin{enumerate}[i)]
\item a morphism of $k$-algebras: $f^\star: \C[V_2] \to \C[V_1] \ , \ h \mapsto h \circ f$
\item a morphism of fields: $f^\star: \Rat(V_2) = \Quot(\O_2) \hookrightarrow \Rat(V_1)=\Quot(\O_1)$ induced by $f^\star$ in i).
\end{enumerate}
\end{Bem}

\begin{conc}
Hence we have the following setting:
\[
\begin{tikzcd}
\O_1 \arrow[r, hook] & \Quot(\O_1)=:K_1\\
\O_2 \arrow[u, hook, "f^\star"] \arrow[r, hook] & \Quot(\O_2)=:K_2 \arrow[u, hook, "f^\star"]
\end{tikzcd}
\quad \quad 
\begin{tikzcd}
\ &V_1=\C \arrow[d, "f"]&\ \\
\C(X) \arrow[u, hook, "f^\star"]&V_2=\Aff^1 \arrow[r, "h"]& \C \\
\end{tikzcd}
\]
\begin{itemize}
\item $\O_1$ and $\O_2$ are Dedekind domains
\item $\Quot(\O_2)=\Quot(\C[X])=\C(X)$\\
$\Quot(\O_1)=\Quot(\C[X,Y]/(f_\lambda)) =\C(X)(\alpha)$ with $\alpha$ has minimal polynomial $Y^2-X(X-1)(X-\lambda) \in \C(X)[Y]$
\item $[K_1:K_2]=2$
\end{itemize}
\end{conc}

Recall: points in $V_1$ resp. $V_2 \leftrightarrow $ maximal ideals in $\O_1$ resp. $\O_2$\\
$f(a)=b \iff f^\star(m_b) \subseteq m_a \iff m_a$ lies above $m_b$.

\[
\begin{tikzcd}
\C[X][\alpha)=\C[X,Y]/(f_\lambda) \arrow[draw=none]{r}[sloped,auto=false]{=} &\O_1 \arrow[r, hook] & \Quot(\O_1)=:K_1 \arrow[draw=none]{r}[sloped,auto=false]{=} & \C(X)(\alpha)\\
(X-b) \subseteq \C[X] \arrow[draw=none]{r}[sloped,auto=false]{=} &\O_2 \arrow[u, hook, "f^\star"] \arrow[r, hook] & \Quot(\O_2)=:K_2 \arrow[u, hook, "f^\star"] \arrow[draw=none]{r}[sloped,auto=false]{=}&\C(X) 
\end{tikzcd}
\]

\begin{Bem}
Consider the prime ideal $(X-b)$ for $b \in \C$.\\
Apply Prop. II.8.15: Observe $\O_1=\O_2[\alpha], f_\alpha(Y)=Y^2-X(X-1)(X-\lambda) \in \O_2[Y]$\\
Consider image $\bar{f_\alpha} \in \O_2/(X-b) [Y] \stackrel{\text{evaluation map}}{\cong} \C [Y]$\\
Hence $\bar{f_\alpha}=Y^2-b(b-1)(b-\lambda) \in \C[Y]=(Y- \sqrt{b(b-1)(b-\lambda)})(Y+\sqrt{b(b-1)(b-\lambda)})$.\\
Prop. II.8.15. $\Rightarrow:$
\begin{itemize}
\item If $b \not \in \{0,1,\lambda\}: o\O_1=\hP_1 \cdot \hP_2$.\\
Hence we have two prime ideals $\hP_1$ and $\hP_2$ lying above $p:=(X-b)$ and the ramification indices are $e_1=e_2=1$.\\
Degree forumula: $2=e_1f_1+e_2f_2 \Rightarrow f_1=f_2=1$.
\item If $b \in \{0,1,\lambda\}: p\O_1=\hP^2$.\\
Hence we have one prime ideal $\hP$ over $p$ with ramification index 2 and $f=1$.
\end{itemize}
\end{Bem}

\begin{Bem}
Why is $f$ in the example $=1$?\\
Recall: $f= \dim_{\O_2/\P} \O_1/\hP$ Here: $\O_2=k[V_2], p=\text{maximal ideal corresp. to a point } b. \Rightarrow \O_2/p\cong \C$. Same way: $\O_1/\hP \cong \C$.
\end{Bem}