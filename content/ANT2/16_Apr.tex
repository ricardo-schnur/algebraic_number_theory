\rhead{16 April 2018}



\begin{proof}
\enquote{(i)} Klar.
\enquote{(ii)} Es genügt zu zeigen, dass $\m$ ein Ideal ist. Dann folgt die Behauptung aus $\m = R \bs R^\times$.
\begin{enumerate}[(a)]
\item Seien $r\in R \bs \{0\}$ und $x\in \m \bs \{0\}$. Angenommen, $rx \not \in\m $. Dann gilt $rx \not\in R$, denn sonst $rx \in R^\times$ und damit auch $x \in R^\times$.
Da $R$ ein Bewertungsring ist, folgt $\frac{1}{rx} \in R$ und damit
\[ \frac{1}{x} = r \frac{1}{rx} \in R
\]
so, das $x \in R^x$ im Widerspruch zur Annahme.
\item Seien $x,y\in \m \bs\{0\}$. Ohne Einschränkung ist $\frac{x}{y} \in R$, da $R$ ein Bewertungsring ist. Mit (a) folgt dann
\[ x+y= \left( 1 + \frac{x}{y} \right) y \in\m.
\]
\end{enumerate}
\enquote{(iii)} Sei $x \in K= \Quot R$ ganz über $R$. Dann existieren $a_0 , \dots, a_{n-1} \in R$
mit
\[ x^n + a_{n-1}x^{n-1} + \cdots + a_0 =0,
\]
also
\begin{align}
x^n =- a_{n-1}x^{n-1} - \cdots - a_0 \tag{$\star$}.
\end{align}
Angenommen, $x \not\in R$. Dann ist $\frac{1}{x} \in R$ und mit ($\star$) folgt der Widerspruch
\[ x = \frac{1}{x^{n-1}} x^n
= -a_{n-1} - a_{n-2} \frac{1}{x} - \cdots - a_o\frac{1}{x^{n-1}} \in R.
\]
\end{proof}

\begin{Kor}
Sei $v\colon K\to \R \cup \{\infty \}$ eine Bewertung mit zugehöriger Norm $\abs{\cdot}$. Dann ist
\[ \O = \left\{ x \in K \,|\, v(x) \geq 0 \right\}
=\left\{ x \in K \,|\, \abs{x} \leq 1 \right\}
\]
ein Bewertungsring mit Einheiten
\[ \O^\times = \left\{ x \in K \,|\, v(x) = 0 \right\}
=\left\{ x \in K \,|\, \abs{x} = 1 \right\}
\]
und eindeutigem maximalem Ideal
\[ \m = \left\{ x \in K \,|\, v(x) > 0 \right\}
=\left\{ x \in K \,|\, \abs{x} < 1 \right\}.
\]
\end{Kor}

\begin{Bsp}
Seien $k$ ein Körper und
\[ k \left\{X\right\} =
\left\{
a_0X^{l_0} + a_1 X^{l_1} + \cdots \,\big|\, a_i \in k, l_i \in \R, l_0 < l_1 < \cdots, \lim_{i \to \infty} l_i = \infty
\right\}
\]
mit Addition und Multiplikation definiert analog zu formalen Laurent-Reihen.
Dann ist $k\{X\}$ ein Körper und $v\colon K\{X\} \to \R \cup \{\infty\}$ mit $0 \mapsto \infty$,
\[ 0 \neq z = a_0X^{l_0} + a_1X^{l_1} + \cdots \mapsto l_0,
\]
wobei $a_0 \neq 0$, ist eine Bewertung, die nicht diskret ist.
\end{Bsp}



\section{Komplettierung}

Sei $K$ stets ein Körper.

\begin{defi}
Sei $(X,d)$ ein metrischer Raum.
\begin{enumerate}[(i)]
\item Eine Folge $(a_n)_n$ von Punkten in $X$ heißt \textbf{Cauchy-Folge}, falls für alle $\varepsilon >0 $ ein $N\in\N$ existiert mit $d(x_n,x_m)< \varepsilon$ für alle $n,m \geq N$.
\item $(X,d)$ heißt \textbf{vollständig}, falls jede Cauchy-Folge in $(X,d)$ konvergiert.
\item $\hat{X}, \hat{d}$ heißt \textbf{Komplettierung} von $(X,d)$, falls
	\begin{itemize}
	\item $(\hat{X},\hat{d})$ ist vollständig,
	\item es existiert eine isometrische Einbettung $\varphi \colon X \hookrightarrow \hat{X}$ mit
	$\varphi(X) \subset \hat{X}$ dicht.
	\end{itemize}
\end{enumerate}
\end{defi}

\begin{Fakt}
Jeder metrische Raum besitzt eine \textbf{eindeutige} metrische Komplettierung.
\end{Fakt}


\begin{Satz}
\begin{enumerate}[(i)]
\item Jeder Körper $(K,\abs{\cdot})$ besitzt eine \textbf{Vervollständigung} $(\hat{K}, \norm{\cdot}, i)$, d.h.
	\begin{itemize}
	\item $(\hat{K}, \norm{\cdot})$ ist ein normierter Körper,
	\item $i \colon K \to \hat{K}$ ist eine Körpererweiterung, die die Norm erhält, d.h.
	$\norm{i(x)} = \abs{x}$ für alle $x \in K$,
	\item $\hat{K}$ ist die metrische Komplettierung von $K$ bezüglich den von den Normen induzierten Metriken.
	\end{itemize}
\item Die Vervollständigung in (i) ist eindeutig wie folgt:
		Sind $(\hat{K}_1, \norm{\cdot}_1, i_1)$, $(\hat{K}_2, \norm{\cdot}_2, i_2)$ Vervollständigungen von $(\hat{K}, \norm{\cdot}, i)$, dann gibt es einen eindeutigen isometrischen Körperisomorphismus $h \colon \hat{K}_1 \to \hat{K}_2$ mit $h\circ i_1 = i_2$.
\end{enumerate}
\end{Satz}


\begin{Fakt}
$\R$ ist die Vervollständigung von $\Q$ bezüglich $\abs{\cdot}_\infty$.
\end{Fakt}

\begin{Prop}
\begin{enumerate}[(i)]
\item $(\Q, \abs{\cdot}_p)$ ist nicht vollständig für $p\neq 2$.
\item Ist $(K,\abs{\cdot})$ eine nicht-archimedische Norm, so gilt:
\[ (a_n)_n \text{ ist Cauchy-Folge in } K \quad \Leftrightarrow \quad 
\abs{a_{n+1} - a_n} \xrightarrow{n \to \infty} 0
\]
\end{enumerate}
\end{Prop}

\begin{proof}
\enquote{(ii)} \enquote{$\Rightarrow$} Klar. \enquote{$\Leftarrow$} Ohne Einschränkung sei $m< n$. Es gilt
\begin{align*}
\abs{a_m-a_n}
=\abs{a_m-a_{m-1} + a_{m-1} - \cdots + a_{n-1} -a_n }
\leq \max \{ \abs{a_{m}-a_{m-1}}, \dots, \abs{ a_{n-1} -a_n }  \}
\end{align*}	
und damit folgt die Behauptung.

\bigskip \enquote{(i)} \textbf{Beweisidee für $p\neq 2$:}
Wähle $a \in  \Z$ mit 
\begin{itemize}
\item[(1)] $p \not | \,a$,
\item[(2)] $a$ kein Quadrat,
\item[(3)] $a\equiv b^2 \mod p$.
\end{itemize}
Konstruiere Folge $(b_n)_n$ mit
\[ b_n \equiv b_{n-1} \mod p^n
\qquad \text{und} \qquad
b_n^2 \equiv a \mod p^{n+1}.
\]
Dann gilt 
\[ \abs{b_n-b_{n-1}}_p \leq \frac{1}{p^n} \xrightarrow{n \to \infty}0,
\]
also ist $(b_n)_n$ eine Cauchy-Folge.
Angenommen $b_n \to b_\infty \in \Q$, dann auch $b_n^2 \to b_\infty^2$.
Andererseits gilt $b_n^2 \to a$, also $a=b_\infty^2$, aber $a$ ist kein Quadrat.

\bigskip Existenz einer solchen Folge $(b_n)$ folgt aus Lemma 5.3.6.
\end{proof}



\begin{Bsp}
$1, 1+p, 1+p+p^2, 1+p+p^2+p^3, \dots$ ist eine Cauchy-Folge. Man kann zeigen, dass diese Folge  in $\Q$ bezüglich $\abs{\cdot}_p$ konvergiert.
\end{Bsp}

\begin{Lem}
Seien $a,b \in\Z, n\in\N,p\neq 2$ prim mit $p\not |\, a$ und $a\equiv b^2\mod p^n$.
Dann existiert ein $x\in\Z$ mit $x\equiv b\mod p^n$ und $x^2\equiv a\mod p^{n+1}$.
\end{Lem}

\begin{proof}
Nach Voraussetzung existiert ein $c\in\Z$ mit $a=b^2+cp^n$.
Wir suchen nun ein $x$ mit $x =b+dp^n$ und $x^2\equiv a\mod p^{n+1}$. Es gilt
\[ (b+dp^n)^2
= b^2+2bdp^n +d^2p^{2n}
= a-cp^n + 2bdp^n +d^2 p^{2n}
\equiv a+p^n (-c+2bd) \mod p.
\]
Wähle also $d$ so, dass $-c+2bd \equiv 0 \mod p$. Dies ist möglich, da $b\not\equiv 0 \mod p$ und $p\neq 2$. Dann leistet $x=b+dp^n$ das Gewünschte.
\end{proof}


\begin{proof}[Beweisidee von Satz 2]
\enquote{(i)} Konstruiere $\hat{K}$ wie folgt:
\begin{itemize}
\item[(1)] $\mathcal{C} = \{ (a_n)_n \, | \, a_n \in K, (a_n)_n \text{ Cauchy-Folge}\}$ mit komponentenweiser Addition und Multiplikation ist ein Ring.
\item[(2)] Sei $\mathcal{N} = \{ (a_n)_n \in \mathcal{C} \,|\, \abs{a_n} \to 0 \}$
			die Menge alle Nullfolgen. Dann ist $\mathcal{N}$ ein Ideal in $\mathcal{C}$, nach Lemma 3.7 sogar maximal.
\item[(3)] Definiere $\hat{K} = \mathcal{C} / \mathcal{N}$ und $i \colon K \hookrightarrow \hat{K}, z \mapsto (z,z,\dots)$.
\item[(4)] Definiere $\norm{(a_n)_n} = \lim_{n\to\infty} \abs{a_n}$.
\end{itemize}
\end{proof}


\begin{Lem}
$\mathcal{N}$ wie oben definiert ist maximal.
\end{Lem}


\begin{proof}
Angenommen, es gibt ein Ideal $I$ mit $\mathcal{C} \supsetneq I \supsetneq \mathcal{N}$. Wähle 
$b=(b_n) \in I \bs \mathcal{N}$, d.h. $b$ ist Cauchy-Folge, aber keine Nullfolge.

\bigskip
\textbf{Erinnerung:}
\begin{itemize}
\item[(1)] $(b_n)$ Cauchy-Folge $\Rightarrow$ $(\abs{b_n})$ Cauchy-Folge
\item[(2)] $(b_n)$ Nullfolge $\Leftrightarrow$ $(\abs{b_n})$ Nullfolge
\end{itemize}

Also existieren $\lambda>0$ und $N\in\N$ so, dass $\abs{b_n}> \lambda$ für alle $n\geq N$.
Definiere $b'= (b_n')$ mit 
\[ b_n' = \begin{cases}
1, &\mbox{} n<N, \\
b_n, &\mbox{} n \geq N.
\end{cases}
\]
Also $b \in \mathcal{C}^\times $und $a= b-b' \in \mathcal{N}\subset I$ so, dass $b' \in I$ und daher $I=\mathcal{C}$.
\end{proof}





