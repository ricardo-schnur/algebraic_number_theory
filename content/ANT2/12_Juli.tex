\rhead{12 Juli 2018}


\begin{proof}
Betrachte die Abbildung $\varphi: G \to \hat{G} \ , \ \sigma \mapsto p_N(\sigma)_N$ mit $p_N: G \to G/N$ ist kanonische Projektion.\\
\underline{Vorüberlegung:} $E$ normaler Zwischenkörper von $L|K$ mit $E|K$ endlich\\
$\Rightarrow N:=N_E:=\Gal(L|E)$ ist zugeh. Normalteiler von endlichem Index. $G/N= \Gal(L|K)/\Gal(L|E) \cong \Gal(E|K)$\\
$\Rightarrow p_N: G \to G/N \cong \Gal(E,K) \ , \ \sigma \mapsto \sigma|E.$\\
\begin{enumerate}[(1)]
\item \underline{Zeige:} $\varphi$ ist injektiv.\\
$\varphi$ ist injektiv $\iff \bigcap \limits_{\substack{ N \trianglelefteq G,\\ [G:N] < \infty}} N = \{1\} \stackrel{Def.}{\iff} G$ ist \underline{\textbf{residuell endlich}}.\\
$\sigma \in \cap N \Rightarrow$ Für $\alpha \in L$ und $E=Z(\alpha)$ gilt: $\sigma \in \Gal(L|E) \Rightarrow \sigma(\alpha)=\alpha \Rightarrow \sigma=id$.
\item $\varphi$ ist surjektiv. Sei $(\sigma_N) \in \hat{G}. [\sigma_N \in G/N = \Gal(E_N|K)$ mit $E_N=L^N]$.\\
Definiere $\sigma : L \to L$ durch: Für $\alpha \in L$ sei $E$ eine endliche Körpererweiterung von $K$ in $L$, die $\alpha$ enthält mit $N:= \Gal(L|E)$. Definiere $\sigma(\alpha):=\sigma_N(\alpha)$. Dies ergibt wohldefinierten Körperhomom., wg. Verträglichkeitsbedingung im Limes $\Rightarrow \sigma \in \Gal(L|K)$ und $\varphi(\sigma)=(\sigma_N)$.
\end{enumerate}
\end{proof}

\begin{Bem}
Damit trägt $\Gal(L|K)$ eine natürliche Topologie als proendliche Komplettierung. $T$ ist die kleinste Topologie, die alle Mengen $\sigma \cdot \Gal(L|E)$ mit $\sigma  \in \Gal(L|K)$ und $E$ endlicher Zwischenkörper von $L|K$ enthält. 
\[\text{ (Induzierte Topologie von: \ } G \cong \hat{G} = \todo{projlim} G/N \subseteq \prod_N G/N \stackrel{p_N}{\to} G/N)\]
Die Menge
\[U_{(1)} := \{\Gal(L|E) \ | \ K \subseteq E \subseteq L \text{ und } E|K \text{ endlich und normal} \}\]
ist Umgebungsbasis von $1=id$ und $\sigma \cdot U_{(1)}$ ist Umgebungsbasis von $\sigma$.
\end{Bem}

\begin{Prop}
Sei $G$ Gruppe mit proendlicher Komplettierung $\hat{G}$.
\begin{enumerate}[(i)]
\item $\hat{G}$ ist Hausdorffsch.
\item $\hat{G}$ ist kompakt.
\end{enumerate}
\end{Prop}

\begin{proof}
\begin{enumerate}[i)]
\item Sei $p_N : \hat{G} \to G/N \ , \ (g_N) \mapsto g_N$ die Projektion auf die Komponente $G/N$.\\
Seien $a=(a_N) \neq b = (b_N)$ in $\hat{G} \Rightarrow \ \exists \ N$ mit $a_N \neq b_N \Rightarrow U_1:= p_N^{-1}(\{a_N\})$ und $U_2:=p_N^{-1}(\{b_N\})$ sind offen und disjunkt und $a \in U_1$ und $b \in U_2$.
\item $\hat{G} \subseteq \prod_N G/N$ und $\prod G/N$ kompakt (Satz von Tychonoff)\\
Zeige also: $\hat{G}$ ist abgeschlossen in $\prod_N G/N$.\\
Für $N_1 \subseteq N_2$ sei
\[A_{N_1, N_2} :=\{a=(a_N) \in \prod_N G/N \ | \ p_{N_2 N_1}(a_{N_1})=a_{N_2}\}.\]
Dann ist 
\[A_{N_1, N_2} = \bigcup \limits_{g \in G/N_2} (p_{N_2}^{-1}(g) \cap \bigcup \limits_{\substack{g' \in G/N_1,\\ p_{N_2, N_1}(g')=g}} p^{-1}_{N_1}(g'))\]
abgeschlossen (endl. Vereinigungen und Urbilder von Punkten hier sind abgeschlossen) und
\[\hat{G} = \bigcap \limits_{\substack{N_1 \subseteq N_2 ,\\ \text{NT von endl. Index}}} A_{N_1, N_2}\]
ist dann ebenfalls abgeschlossen.
\end{enumerate}
\end{proof}

\begin{Bem}
$G$ topologische Gruppe, $U$ Untergruppe von $G$. $U$ offen $\Rightarrow U$ abgeschlossen.
\end{Bem}

\begin{proof}
$G \setminus U = \bigcup_{\sigma \in G \setminus U} \sigma \cdot U$ ist offen, da mit $U$ auch alle $\sigma \cdot U$ offen sind.
\end{proof}

\begin{Prop}
$L|K$ galoissche Körpererweiterung.
\begin{enumerate}[i)]
\item $E|K$ endlich $\Rightarrow \Gal(L|E)$ offene Untergruppe von $\Gal(L|K)$ und damit auch abgeschlossen nach Bem. 1.6.
\item $F|K$ beliebige Körpererweiterung $\Rightarrow \Gal(L|F)$ ist abgeschlossen.
\end{enumerate}
\end{Prop}

\begin{proof}
\begin{enumerate}[i)]
\item folgt aus Bem. 2.1.6
\item $\Gal(L|F) = \bigcap \limits_{\substack{K \subseteq E \subseteq F,\\ [E:K] \leq \infty}} \Gal(L|E)$. Dann wende i) an.
\end{enumerate}
\end{proof}

\begin{Satz}
Die Abbildungen $\phi$ und $\psi$ aus Erinnerung 1.3 definieren eine Bijektion zwischen den Zwischenkörpern von $L|K$ und den abgeschlossenen $UG$ von $\Gal(L|K)$.
\end{Satz}

\begin{proof}
Zu zeigen ist, dass $\psi \circ \phi = id$, d.h.:\\
$\Gal(L|L^H)=H$ für abgeschlossene Untergruppen $H$.\\
Wir zeigen sogar: $H$ Untergruppe von $\Gal(L|K) \Rightarrow \Gal(L|L^H)=\overline{H}$ mit $\overline{H}=$ Abschluss von $H$.
\begin{enumerate}[(1)]
\item $H \subseteq \Gal(L|L^H) \Rightarrow \overline{H} \subseteq \Gal(L|L^H)$, da $\Gal(L|L^H)$ abgeschlossen.
\item Zeige: $\Gal(L|L^H) \subseteq \overline{H}$. Sei $\sigma \in \Gal(L|L^H)$.\\
\underline{Idee:} Verwende: $\{\sigma \cdot \Gal(L|E) \ | \ E|K \text{ endliche Körpererweiterung mit } E \subseteq L \}$ ist Umgebungsbasis von $\sigma$.\\
\underline{Zeige also:} $\forall K \subseteq E \subseteq L$ mit $E|K$ endlich + normal: $\sigma \cdot \Gal(L|E) \cap H \neq \emptyset.$
%\begin{tikzcd}
%&&L^H \cdot E&&\\
%\\
%L^H \arrow{rruu}{endlich} &&&& E \arrow[dashed]{lluu}\\
%&&& \arrow[Rightarrow]{lu}\\
%&&E \arrow[dashed]{lluu} \arrow{rruu}{endlich}
%\end{tikzcd}
Verwende weiter: $\exists \ F: L^H \subseteq F \subseteq L$ mit $F|L^H$ endlich und normal und $F \supseteq E$\\
$\Rightarrow \Gal(L|F) \subseteq \Gal(L|E)$\\
$\Rightarrow \{\sigma \cdot \Gal(L|F) \ | \ F|L^H \text{endlich + normal}\}$ ist ebenfalls Umgebungsbasis von $\sigma$.\\
Zeige also: $\forall \ F$ wie oben gilt: $\sigma \cdot \Gal(L|F) \cap H \neq \emptyset$.\\
\underline{Vorüberlegung:} \begin{itemize}
\item $H \subseteq \overline{H} \Rightarrow L^H \supseteq L^{\overline{H}}$
\item (1) $\Rightarrow L^{\overline{H}} \supseteq L^H$
\end{itemize}
$\Rightarrow L^{\overline{H}}=L^H.$
Betrachte für $F$ wie oben die Abbildung
\begin{align*}
\varphi: &\overline{H} \to \Gal(F|L^H)\\
&\tau \mapsto \tau|F
\end{align*}
Für die Untergruppe $\Bild(\varphi) \subseteq \Gal(F |L^H)$ gilt:
\[F^{\Bild(\varphi)} = L^H.\]
Hauptsatz der Galois.Theorie für endliche Körpererweiterungen:
\[\Bild(\varphi)=\Gal(F|L^H)\]
Also ist $\varphi$ surjektiv: $\Rightarrow \exists \ \tau \in \overline{H}$ mit $\tau|_F = \sigma|_F \Rightarrow \sigma^{-1}\circ \tau \in \Gal(L|F) \Rightarrow \tau \in \sigma \cdot \Gal(L|F)$. Also: $\tau \in \overline{H} \cap \sigma \Gal(L|F)$.
\end{enumerate}
\end{proof}