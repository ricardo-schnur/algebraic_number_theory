% !TeX spellcheck = de_DE
\rhead{16 Juli 2018}


\section{Galois-Theorie für Bewertungen}


\begin{setting}
	In diesem Abschnitt sei $L/K$ eine Galois-Erweiterung, $G= \Gal(L/K)$ und $v$ eine Bewertung auf $K$. $G$ operiert auf
	\[ \M_v = \left\{
	w; \, \text{$w$ Bewertung auf $L$, $w/v$}
	\right\}
	\]
	durch $\sigma \colon w \mapsto w \circ \sigma$.
\end{setting}

\begin{Prop}[Approximationssatz]
	Seien $K$ ein Körper und $\abs{\cdot}_1, \dots, \abs{\cdot}_n$ paarweise nicht-äquivalente Normen auf $K$.
	Sind $a_1,\dots, a_n \in K$ so existiert für jedes $\varepsilon>0$ ein $x \in K$ mit
	\[ \abs{x-a_i}_i < \varepsilon
	\]
	für alle $i=1,\dots,n$.
\end{Prop}

\begin{proof}
Siehe Neukirch, II (3.4).
\end{proof}

\begin{Prop}
	$G=\Gal(L/K)$ operiert transitiv auf 
	\[ \M_v = \left\{
	w; \, \text{$w$ Bewertung auf $L$, $w/v$}
	\right\}.
	\]
\end{Prop}

\begin{proof}
	\textbf{Schritt 1:} Zeige die Behauptung für $L/K$ endlich.
	
	\bigskip
	Annahme: Es existieren $w,w' \in \M_v$ die nicht in der gleichen Bahn liegen, d.h.
	\[ Gw = \left\{  w\circ\sigma; \, \sigma \in G \right\}
	= \left\{ w_1, \dots, w_r \right\}
	\]
	und
	\[ Gw' = \left\{  w'\circ\sigma; \, \sigma \in G \right\}
	= \left\{ w_1', \dots, w_s' \right\}
	\]
	sind disjunkt.
	
	
	\bigskip
	Der Approximationssatz (Proposition 2.2) angewendet auf $w_1,\dots, w_, w_1', \dots, w_s'$ und
	$a_1= \cdots = a_r =0$, $a_{r+1} = \dots = a_{r+s} = b$ mit $\abs{b}>1$ liefert ein $x\in L$ mit
	\[ \abs{x}_{w_i} < 1
	\]
	für $i=1,\dots, r$ und
	\[ \abs{x}_{w_j'} > 1
	\]
	für $j=1,\dots, s$. Also gilt für alle $\sigma \in G$, dass
	\[ \abs{\sigma(x)}_{w} <1
	\quad \text{und} \quad
	 \abs{\sigma(x)}_{w'} >1.
	\]
	Für $\alpha \in \Norm_{L/K} (x) = \prod_{\sigma \in G} \sigma(x)$ gilt:
	\begin{align*}
	\abs{\alpha}_v
	&= \abs{\alpha}_w 
	= \prod_{\sigma \in G} \abs{\sigma(x)}_w
	<1 \\
	\abs{\alpha}_v
	&= \abs{\alpha}_{w'}
	= \prod_{\sigma \in G} \abs{\sigma(x)}_{w'}
	>1
	\end{align*}
	Dies ist ein Widerspruch, also besteht $\M_v$ nur aus einer Bahn.
	
	\bigskip \textbf{Schritt 2:} Die Behauptung gilt auch für unendliche Körpererweiterungen $L/K$.
	
	Seien $w,w' \in \M_v$. Für $K \subset E \subset L$ mit $E/K$ endlich und galoissch definiere
	\[ X_E = \left\{
	\sigma \in G; \, w\circ\sigma|_E = w'|_E
	\right\}
	\subset G.
	\]
	Nach Schritt 1 ist $X_E \neq \emptyset$. $X_E$ ist abgeschlossen, denn für $\sigma \not \in X_E$ gilt
	\[ \sigma \Gal(L/E) \cap X_E = \emptyset
	\]
	und damit ist
	\[ \sigma \Gal(L/E) \subset G \bs X_E
	\]
	so, dass $G \bs X_E$ offen ist.
	
	\bigskip
	Noch zu zeigen: 
	\[ \bigcap_{\substack{K \subset E\subset L \\ [E/K]< \infty}} X_E \neq \emptyset
	\]
	
	Annahme: 
	\[ \bigcap_{\substack{K \subset E\subset L \\ [E/K]< \infty}} X_E = \emptyset
	\]
	Da $G$ kompakt ist, existieren $E_1, \dots, E_M$ mit
	\[ X_{E_1} \cap \cdots \cap X_{E_m} = \emptyset.
	\]
	Es gilt aber für
	\[ E = E_1 \cdots E_M,
	\]
	dass $E/K$ endlich ist und 
	\[ X_E = X_{E_1} \cap \cdots \cap X_{E_m} = \emptyset.
	\]
	Dies widerspricht Schritt 1.
\end{proof}

\begin{defi}
	Die \textbf{Zerlegungsgruppe} von $w/v$ ist gegeben durch
	\begin{align*}
	G_w
	=G_w(L/K)
	&= \left\{ \sigma \in \Gal(L/K); \, w\circ\sigma = w \right\} \\
	&= \Stab_{\Gal(L/K)}(w).
	\end{align*}
\end{defi}

\begin{Bem}
	Für $\sigma \in G_w$ gilt
	\[ w(\sigma(\alpha)) = w(\alpha)
	\]
	für alle $\alpha \in L$ so, dass $\sigma(\O_w) = \O_w$ und $\sigma(p_w) = p_w$.
	Dies liefert einen Homomorphismus
	\[ \Phi \colon G_w \to \Gal(\kappa_w/\kappa_v).
	\]
\end{Bem}

\begin{defi} Die Gruppe
	\begin{align*}
	I_w 
	=I_w(L/K)
	&= \ker(\phi)= \left\{
	\sigma \in G_w; \, \sigma(x) \equiv x \mod p_w \text{ für alle } x \in \O_w
	\right\}
	\end{align*}
	heißt \textbf{Trägheitsgruppe}.
\end{defi}


\begin{Prop}
	$G_w$ und $I_w$ sind abgeschlossene Untergruppe von $\Gal(L/K)$.
\end{Prop}


\begin{proof}
	Sei $\sigma \in \overline{G_w}$, dem Abschluss von $G_w$. Für alle Zwischenkörper $E$ von $L/K$ mit $E/K$ endlich existiert dann ein $\sigma_E \in G_w$ mit
	$\sigma_E \in \sigma \Gal(L/E)$, d.h.
	\[ \sigma_E|_E = \sigma|_E.
	\]
	Also gilt, da $\sigma_E \in G_w$,
	\[ w(\sigma(\alpha)) = w(\alpha)
	\]
	für alle $\alpha \in E$.
	
	\bigskip
	Der Beweis für $I_w$ geht analog.
\end{proof}

\begin{Prop}
	\begin{align*}
	G_w
	&=
	\left\{ 
	\sigma \in \Gal(L/K); \,\text{$\sigma$ ist stetig bezüglich der Topologie, die $w$ auf $L$ induziert}
	\right\}
	\end{align*}
\end{Prop}

\begin{proof}
	\enquote{$\subset$} Da $\sigma \in G_w$, erhält $\sigma$ die von $w$ induzierte Norm $\abs{\cdot}_w$ so, dass $\sigma$ stetig ist.
	
	\bigskip \enquote{$\supset$} Sei $\sigma$ stetig. Sei $\abs{x}_w < 1$, also ist $(x^n)$ eine Nullfolge. Damit ist auch $(\sigma(x)^n)$ eine Nullfolge, also
	$\abs{\sigma(x)}_w <1$ und damit
	\[ \abs{x}_{w\circ \sigma} < 1.
	\]
	Daraus folgt aus dem Beweis von I.1.9, dass $w$ und $w\circ\sigma$ äquivalent und damit als gemeinsame Fortsetzung von $v$ sogar gleich sind.
\end{proof}


\begin{Satz}
	Sei $L/K$ eine galoissche Körpererweiterung, $v$ eine Bewertung auf $K$ und $w$ eine Fortsetzung auf $L$. Weiter seien $K_v$ und $L_w$ die jeweiligen Lokalisierungen mit den Bewertungen $\hat{v}$ beziehungsweise $\hat{w}$. Dann gilt
	\begin{enumerate}[(i)]
		\item $G_w(L/K) \cong \Gal(L_w/K_v)$
		\item $I_w(L/K) \cong I(L_w/K_v)$
		\item $\kappa_{\hat{w}} / \kappa_{\hat{v}}$ und $\kappa_w / \kappa_v$ sind galoissch mit
		\[ \Gal \left( \kappa_{\hat{w}} / \kappa_{\hat{v}} \right)
		\cong \Gal \left( \kappa_w / \kappa_v \right)
		\]
	\end{enumerate}
\end{Satz}

\begin{proof}
	\enquote{(i)} Sei
	\[ h \colon \Gal(L_w/K_v) \to \Gal(L/K), \, \sigma \mapsto \sigma|_L.
	\]
	Beachte: Da $K_v$ und $L_w$ henselsch sind gilt
	\[ \Gal(L_w/K_v)=G_{\hat{w}}(L_w/K_v).
	\]
	Also gilt $\sigma \in \Gal(L_w/K_v)$ und damit ist $\sigma$ stetig bezüglich der von $\hat{w}$ erzeugten Topologie und somit $\sigma|_L \in G_w(L/K)$.
	Insgesamt folgt
	\[ \Bild (h) \subset G_w(L/K).
	\]
	Jede stetige Abbildung $L\to L$ lässt sich auf $L_w$ fortsetzen und zwar auf eindeutige Weise.
	Daher ist
	\[ h \colon \Gal(L_w/K_v) \to G_w(L/K)
	\]
	ein Isomorphismus.
	
	\bigskip \enquote{(iii)}
	Nach Proposition I.5.2 gilt $\kappa_{\hat{w}} \cong \kappa_w$ und $\kappa_{\hat{v}} \cong \kappa_v$. Nach AZT1, Bemerkung II.9.12 ist $\kappa_{\hat{w}} / \kappa_{\hat{v}}$ galoissch.
	
	\bigskip \enquote{(ii)}
	Betrachte folgendes kommutatives Diagramm:
	\[ \begin{tikzcd}
	\Gal(L_w/K_v)
		\arrow{r}{h^{-1}}
		\arrow[swap]{r}{\cong}
		\arrow{d}
	& G_w(L/K)
		\arrow{d}
	\\
	\Gal\left(  \kappa_{\hat{w}} / \kappa_{\hat{v}} \right)
		\arrow[swap]{r}{\cong}
	& \Gal(\kappa_w / \kappa_v)
	\end{tikzcd}
	\]
	Damit $I(L_w/K_v) \cong I_w(L/K)$.
\end{proof}

\begin{Prop}
	Sei nun $v$ eine diskrete Bewertung und $L/K$ eine endliche Körpererweiterung. Dann gilt:
	\begin{enumerate}[(i)]
		\item $1\to I_w(L/K)\to G_w \to \Gal(\kappa_w/\kappa_v) \to 1$ ist eine kurze exakte Sequenz.
		\item $f(w/v) = \# \Gal(\kappa_w/\kappa_v)$ und $e(w/v) = I_w(L/K)$
		\item Seien
		\[ T_w = T_w(L/K) = L^{I_w(L/K)}
		\]
		der \textbf{Trägheiskörper} und
		\[ Z_w=Z_w(L/K) = L^{G_w(L/K)}
		\]
		der \textbf{Zerlegungskörper}. Erhalte Körperkette
		\[ K \subset Z_w \subset T_w \subset L
		\]
		mit maximal unverzweigtem Zwischenkörper $T_w/Z_w$ von $L/Z_w$.
		\item Falls es auf $L$ nur genau eine Fortsetzung von $v$ gibt (z.B. falls $(K,v)$ henselsch ist), dann gilt $G_w(L/K) = \Gal(L/K)$ und $Z_w =K$.
	\end{enumerate}
\end{Prop}

\begin{proof}
	\enquote{(i)} Folgt aus Theorem 11 und AZT1, Bemerkung II.9.12.
	
	\bigskip \enquote{(ii)} Folgt aus
	\begin{align*}
	\G_w
	&= \# \Gal(\kappa_w / \kappa_v) \cdot \# I_w(L/K)
	=f(w/v) \cdot \#I_w(L/K)
	\end{align*}
	und
	\begin{align*}
	\G_w
	&= \# \Gal(L_w / K_v) 
	= e(\hat{w}/\hat{v}) f(\hat{w}/\hat{v}) 
	= e(w/v) f(w/v). 
	\end{align*}
	
	\bigskip \enquote{(iii)} Zeige die Behauptung zunächst für $K$ henselsch und somit $K=Z_w$:
	
	\bigskip \textbf{(1)} $ \Gal(T_w/K) \cong \Gal(L/K) / \Gal(L/T_w)
	 \cong \Gal(L/K) / I_w(L/K) \cong \Gal(\kappa_w/\kappa_v)$
	
	\bigskip \textbf{(2)} Sei $M$ der maximal unverzweigte Zwischenkörper von $L/K$ und $w_M=w|_M$. $M/K$ ist normal, denn für $\sigma\in\Gal(L/K)$ gilt, dass $\sigma(M)$ ebenfalls unverzweigt ist und damit $\sigma(M)\subset M$. (i) angewendet auf $M/K$ ergibt:
	\[ 1 \to I_{w_M}(M/K) \to \Gal(M/K) \to \Gal(\kappa_{w_M} / \kappa_v) \to 1
	\]
	Außerdem impliziert $M/K$ unverzweigt, dass
	\[ \# I_{w_M}(M/K) = e(M/K) =1
	\]
	und somit gilt $\Gal(M/K) \cong \Gal (\kappa_{w_M} / \kappa_v)$.
	
	\bigskip \textbf{(3)} Aus (1) und (2) erhält man
	\[ \Gal(L/T_w)=\Gal(L/M)= \ker \Phi = I_w(L/K).
	\]
	
	\bigskip Sei nun möglicherweise $K$ nicht henselsch.
	Da
	\[ G_w = \{ \sigma \in \Gal(L/K); \, \sigma \text{ stetig} \}
	\]
	folgt, dass $ K_v = L_w^{\Gal(L_w/K_v)} $ die Vervollständigung von $Z_w = L^{\sigma_w}$ is und $T_{\hat{w}} = L_w^{G_{\hat{w}}(L_w/K_v)}$ die Vervollständigung von $T_w = L^{I_w(L/K)}$ ist.
	Es gilt damit $e(T_w/Z_w)=e(T_{\hat{w}}/K_v) =1$. Außerdem gilt für den maximal unverzweigten Zwischenkörper $M$ von $T_w/Z_w$, dass $M/Z_w$ galoissch ist.
	(i) angewendet auf $M/Z_w$ und $T_w/Z_v$ ergibt
	\[ G_{w_M} \cong \Gal(\kappa_w/\kappa_v) \cong G_{w_\sigma}.
	\]
	Damit $M= T_w$.
\end{proof}












