\rhead{18 April 2018}

\begin{Lem}
Sei $a=(a_n)\in\mathcal{C}$. Dann existiert $\norm{a} = \lim \abs{a_n}$, ist invariant unter Addition von Nullfolgen und $a\mapsto \norm{a}$ definiert eine Norm auf $\hat{K}$.
\end{Lem}

\begin{proof}
Ist $a=(a_n)$ eine Cauchy-Folge, so auch $(\abs{a_n})$, also existiert $\lim \abs{a_n}$, da $\R$ vollständig ist.
Ist weiter $(c_n)$ eine Nullfolge so gilt
\[ \norm{a} \xleftarrow{n} \abs{a_n} - \abs{c_n} \leq \abs{a_n+c_n} \leq \abs{a_n} + \abs{c_n} \xrightarrow{n} \norm{a}
\]
so, dass $\lim \abs{a_n+c_n} = \norm{a}$. Schließlich gilt
\[ \norm{a} = 0 \quad \Leftrightarrow \quad \lim\abs{a_n} = 0 \quad \Leftrightarrow \quad
(a_n) \text{ Nullfolge}
\]
und die restlichen Normeigenschaften bleiben bei Limesbildung erhalten.
\end{proof}

\begin{Lem}
$i(K)$ liegt dicht in $\hat{K}$.
\end{Lem}

\begin{proof}
Sei $a=(a_n)\in\hat{K}$ und definiere die Folge $a^{(n)} \in \hat{K}$ durch 
\[ a^{(n)} = i(a_n) = (a_n, a_n , \dots).
\]
Dann gilt $a^{(n)} \xrightarrow{n} a$, denn $(a_n)$ ist eine Cauchy-Folge, also existiert zu allen $\varepsilon>0$ ein $N\in\N$ mit $\abs{a_n-a_m}< \varepsilon$ für alle $n,m \geq N$ so, dass
\[ \norm{a-a^{(n)}} = \lim\limits_{j \to\infty} \abs{a_j-a_n}< \varepsilon
\]
für alle $n \geq N$ gilt.
\end{proof}

\begin{Lem}
$\hat{K}$ ist vollständig.
\end{Lem}


\begin{proof}
Sei $(a^{(n)})$ eine Cauchy-Folge in $\hat{K}$. Nach Lemma 5.3.8 existiert für alle $n\in\N$ ein $x_n\in K$ mit $\norm{i(x_n)-a^{(n)}}< \frac{1}{n}$, das heißt
\[ \lim\limits_{j\to\infty} \abs{x_n-a_j^{(n)}} < \frac{1}{n}.
\]
Definiere $x=(x_n)$ und zeige:
\begin{itemize}
\item[(1)] $x$ ist Cauchy-Folge in $K$ und somit $x \in \hat{K}$.
\item[(2)] $a^{(n)} \xrightarrow{n} x$ in $\hat{K}$.
\end{itemize}
\enquote{(1)} Sei $\varepsilon >0$. Dann existiert $N_1$ mit $\norm{a^{(n)}-a^{(m)}}< \frac{\varepsilon}{2}$ für alle $N,m \geq N_1$. Wähle $N_1$ so groß, dass auch 
\[ \frac{1}{n} + \frac{1}{m} < \frac{\varepsilon}{2}
\]
für alle $N,m \geq N_1$ gilt. Dann folgt
\begin{align*}
\abs{x_n-x_m}
=\norm{i(x_n)-i(x_m)}
&\leq \norm{i(x_n)-a^{(n)}} + \norm{a^{(n)-a^{(m)}}} + \norm{a^{(m)-i(x_m)}} \\
&<\frac{1}{n} + \frac{\varepsilon}{2} + \frac{1}{m}
< \varepsilon.
\end{align*}

\enquote{(2)} Sei $\varepsilon>0$. Dann existiert $N_1$ mit $\abs{x_n-x_m} < \frac{\varepsilon}{2}$  für alle $n,m \geq N_1$. Wähle $N_2 \geq N_1$ mit $\frac{1}{N_2} < \frac{\varepsilon}{2}$. Dann gilt
\begin{align*}
\norm{a^{(n)}-x}
=\lim\limits_{j\to\infty} \abs{a_j^{(n)}-x_j}
\leq \lim\limits_{j\to\infty} \left( \abs{a_j^{(n)}-x_n} + \abs{x_n-x_j} \right)
\leq \frac{1}{n} + \frac{\varepsilon}{2}
< \varepsilon
\end{align*}
für alle $n\geq N_2$, also ist $\hat{K}$ vollständig.
\end{proof}



\begin{Lem}
In der Situation von Theorem 2 (ii) gilt: Definiere $h\colon \hat{K}_1 \to \hat{K}_2$ durch:
\[ a = \lim\limits_{n \to \infty} i_1(a_n) \text{ mit } a_n \in K \mapsto 
h(a) = \lim\limits_{n \to \infty} i_2(a_n)
\]
\end{Lem}

\begin{proof}
Nachrechnen.
\end{proof}

\begin{proof}[Beweis von Theorem 2]
Folgt aus Lemma 3.7-3.11.
\end{proof}

\begin{Bem}
Es sei $(\hat{K}, \norm{\cdot})$ die Vervollständigung von $(K, \abs{\cdot})$ für eine nicht-archimedische Norm $\abs{\cdot}$ und 
\[ \hat{v} \colon \hat{K} \to \R\cup\{\infty \}, \, x \mapsto - \log \norm{x}
\]
die von $\norm{\cdot}$ induzierte Bewertung. Dann gilt:
\begin{enumerate}[(i)]
\item Für $x=(c_n) \in\hat{K}$ gilt $\hat{v}(x) = \lim\limits_{n\to\infty} v(c_n)$, wobei $v = - \log\abs{\cdot}$.
\item Für $x\neq 0$ wird $v(c_n)$ stationär.
\end{enumerate}
\end{Bem}

\begin{proof}
\enquote{(i)} Ergibt sich aus den Definitionen.
\enquote{(ii)} Für große $n$ wird $\norm{x-c_n}$ klein, also wird $\hat{v}(x-c_n)$ groß.
Nach Bemerkung 2.3 gilt daher
\[ \hat{v}(c_n) = \hat{v}(c_n-x+x) = \min \{ \hat{v}(c_n-x), \hat{v}(x) \} = \hat{v}(x).
\]
Also wird die Folge $(v(c_n))$ stationär.
\end{proof}


\begin{defi}
Die Vervollständigung von $\Q$ bezüglich $\abs{\cdot}_p$ heißt der \textbf{Körper der $p$-adischen Zahlen} $\Q_p$. Bezeichne die von $\abs{\cdot}_p$ auf $\Q_p$ induzierte Norm ebenfalls mit 
$\abs{\cdot}_p$ und definiere den zugehörigen Bewertungsring durch
\[ \Z_p = \{  q \in \Q_p \,|\, \norm{q}_p \leq 1 \}.
\]
Wir nennen $\Z_p$ den \textbf{Ring der ganzen $p$-adischen Zahlen}.
\end{defi}


\section{Projektive Limiten}
Sei $\Ccat$ eine Kategorie. Bezeichne für zwei Objekte $X, Y \in \Ccat$ mir $\Mor(X,Y)$ die Menge der Morphismen von $X$ nach $Y$.

\begin{defi}
Ein \textbf{inverses System} besteht aus
\begin{enumerate}[(i)]
\item einer partiell geordneten Menge $(I,\leq)$,
\item einem Objekt $A_i\in\Ccat$ für jedes $i\in I$,
\item einem Morphismus $f_{ij} \colon A_j \to A_i$ für jedes $(i,j) \in I \times I$ mit $i\leq j$ mit folgenden Verträglichkeitseigenschaften:
\begin{itemize}
\item $f_{ii} = \id_{A_i}$
\item $f_{ik} = f_{ij} \circ f_{jk}$, falls $i \leq j \leq k$
\end{itemize}
\end{enumerate}
\end{defi}


\begin{Bsp}
Sei $\Ccat=\Groups$.
\begin{enumerate}[(i)]
\item Sei $I=\N$ mit  $n\leq m$ falls $n |m$. Dann ist
$A_n = \Z / n\Z$ und für $n \leq m$ ist
\[ f_{nm} \colon \Z / m\Z \to \Z / n\Z
\]
die natürliche Projektion.
\item Allgemeiner sei $G$ eine beliebige Gruppe und
\[I = \{ N \, | \, N \text{ Normalteiler von } G \text{ von endlichem Index}   \}
\]
mit $N_q \leq N_2$ falls $N_2 \subset N_1$. Dann ist $A_N = G /N$ und
\[ f_{NM} \colon G/M \to G/N
\]
die natürliche Projektion.
\end{enumerate}
\end{Bsp}

\begin{defi}
Der \textbf{projektive Limes} oder \textbf{inverse Limes} eines inversen Systems wie in Definition 5.4.1 besteht aus
\begin{enumerate}[(i)]
\item einem Objekt $A$,
\item einem Morphismus $\Pi_i \colon A \to A_i$ für jedes $i \in I$,
\end{enumerate}
so, dass
\begin{itemize}
\item[(A)] $f_{ij} \circ \Pi_j = \Pi_i$ für jedes $(i,j) \in I \times I$ mit $i \leq j$,
\item[(B)] für jedes Objekt $B$ mit einer Familie $\{ \Psi_i \colon B \to A_i \,|\,i\in I \}$ die ebenfalls (A) erfüllen gibt es einen eindeutigen Morphismus $u\colon B \to A$ so, dass $\Pi_i \circ u = \Psi_i$ für alle $i \in I$ gilt.
\end{itemize}
Der projektive Limes $(A, \{\Pi_i \}_{i\in I})$ wird mit $\limproj_{i\in I} A_i$ bezeichnet.
\end{defi}

\begin{Bem}
Falls ein projektiver Limes wie in Definition 5.4.3 definiert existiert, dann ist er eindeutig bis auf eindeutigen Isomorphismus, das heißt:
Ist $(A', \{\Pi_i' \}_{i\in I})$  ebenfalls ein projektiver Limes, dann gibt es einen eindeutigen Isomorphismus $u\colon A' \to A$ mit $\Pi_i\circ u = \Pi_i'$ für alle $i \in I$.
\end{Bem}

\begin{proof}
Wie immer.
\end{proof}

\begin{Prop}
In der Kategorie $\Groups$ gibt es für jedes inverse System einen projektiven Limes.
\end{Prop}

\begin{proof}
Verwende die Notation aus Definition 5.4.1. Definiere
\[ A = \left\{
a=(a_i)_{i\in I} \in \prod_{i\in I}A_i \, \Big| \, f_{ij}(a_j) = a_i \text{ für alle } i \leq j
\right\}
\]
und $\Pi_i \colon A \to A_i, \, a \mapsto a_i$. Mit komponentenweiser Verknüpfung wird $ \prod_{i\in I}A_i$ zu einer Gruppe und $A$ ist eine Untergruppe. Zeige nun, dass $A$ projektiver Limes ist. 
Offensichtlich gilt Eigenschaft (A), zu (B) definiere 
\[ u \colon B \to A, \, b \mapsto (\Psi_i(b_i))_{i\in I}.
\]
\end{proof}