% !TeX spellcheck = de_DE
\rhead{11 Juni 2018}

\begin{proof}
	\begin{enumerate}[i)]
		\item \OE \ $L|K$ endlich $\Rightarrow \kappa_w| \kappa_v$ endlich. Außerdem: $\kappa_w|\kappa_v$ separabel $\Rightarrow \ \exists \ \overline{\alpha} \in \kappa_w$ mit $\kappa_w=\kappa_v(\overline{\alpha})$. Seien $\alpha \in \O_w$ ein Lift von $\overline{\alpha}$ und $f_\alpha(X) \in \O_v[X]$ das Minimalpolynom von $\alpha \stackrel{Lem.\ 9.4}{\Longrightarrow} L=K(\alpha)$ und Bild $\overline{f_\alpha} \in \kappa_v[X]$ von $f_\alpha$ ist Minimalpolynom von $\overline{\alpha} \Rightarrow L'=K'(\alpha)$.\\
		Sei $g_\alpha \in \O_{v'}[X]$ Minimalpolynom von $\alpha'$ in $L'|K' \Rightarrow g_\alpha$ ist Teiler von $f_\alpha \Rightarrow \overline{g_\alpha}$ ist Teiler von $\overline{f_\alpha}$ und damit separabel. Bsp. 9.3 $\Rightarrow L'|K'$ unverzweigt.
		\item Wiederum dürfen wir von endlichen Körpererweiterungen ausgehen. Die Kette $K \subset L \subset L'$ induziert die Kette $\kappa_v \subset \kappa_w \subset \kappa_{w'}$ der Restklassenkörper.
		Aus der Algebra ist bekannt, dass $\kappa_{w'}$ über $\kappa_v$ separabel ist genau dann, wenn 
		$\kappa_{w'}$ über $\kappa_w$ und $\kappa_{w}$ über $\kappa_v$ separabel sind.
		Verwende weiterhin:
		
		\textbf{(1)} $[L':K] = [L':L] [L:K]$
		
		\textbf{(2)} $[\kappa_{w'}:\kappa_{v}] = [\kappa_{w'}:\kappa_{w}] [\kappa_{w}:\kappa_{v}]$
		
		\bigskip \enquote{$\Rightarrow$} Anwenden von i) auf
		\[\begin{tikzcd}
		& L' & \\
		L' 
		\arrow[hook]{ur}{\id}
		& & L 
		\arrow[hook, swap]{ul}{\text{unverzweigt}}
		\\
		& K 
		\arrow[hook]{ur}
		\arrow[hook]{ul}{\text{unverzweigt}}
		&
		\end{tikzcd}
		\]
		zeigt, dass $L'$ über $L$ unverzweigt ist und somit $[L':L] = [\kappa_{w'}:\kappa_{w}]$. Damit folgt aus (1) und (2), dass $[L':K] = [\kappa_{w'}:\kappa_{v}]$
		
		\bigskip \enquote{$\Leftarrow$} Folgt aus (1) und (2).
	\end{enumerate}
\end{proof}


\begin{Kor}
	Seien $L_1 /K$ und $L_2 /K$ algebraische Körpererweiterungen. Es sind genau dann $L_1/K$ und $L_2/K$ unverzweigt, wenn $L_1L_2/K$ unverzweigt ist.
\end{Kor}


\begin{proof}
	\enquote{$\Rightarrow$} Folgt aus:
	\[\begin{tikzcd}
	& L' & \\
	L' 
	\arrow[hook]{ur}{\text{unverzweigt}}
	& & L 
	\arrow[hook, swap]{ul}{\text{unverzweigt}}
	\\
	& K 
	\arrow[hook, swap]{ur}{\text{unverzweigt}}
	\arrow[hook]{ul}{\text{unverzweigt}}
	&
	\end{tikzcd}
	\]
	
	\enquote{$\Leftarrow$} Folgt aus Proposition 9.5.(ii).
\end{proof}


\begin{Bemdef}
	Sei $L/K$ eine algebraische Körpererweiterung und
	\[ S = \left\{
	E; \, E \text{ ist Zwischenkörper von $L/K$ und $E/K$ is unverzweigt}.
	\right\}
	\]
	Dann hat $S$ bezüglich $\subset$ ein maximales Element $T$. 
	$T$ heißt der \textbf{maximale unverzweigte Zwischenkörper von $L/K$} und erhält jeden unverzweigten  Zwischenkörper $E$ von $L/K$.
\end{Bemdef}

\begin{proof}
	Lemma von Zorn und Korollar 9.6.
\end{proof}

\begin{defi}
	Sei $\overline{K}$ der algebraische Abschluss von $K$. Der maximale unverzweigte Zwischenkörper $K^{\nf}$ von $\overline{K} /K$ heißt \textbf{maximale unverzweigte Körpererweiterung von $K$}.
\end{defi}

\begin{Prop}
	Sei $L/K$ eine algebraische Körpererweiterung und $T$ der maximal unverzweigte Zwischenkörper mit Bewertung $w_T$. Dann ist
	\begin{align*}
	\kappa_{w_T}
	&= \left\{
	\overline{\alpha} \in \kappa_w; \, \text{$\overline{\alpha}$ ist separabel über $\kappa_v$}
	\right\}
	\end{align*}
	der separable Abschluss von von $\kappa_v$ in $\kappa_w$.
\end{Prop}


\begin{proof}
	Die Inklusion $K\subset T \subset L$ induziert $\kappa_v \subset \kappa_{w_T} \subset \kappa_v$.
	
	\bigskip \enquote{$\subset$} Aus $T/K$ unverzweigt folgt $\kappa_{w_T}$ über $\kappa_v$ separabel.
	
	\bigskip \enquote{$\supset$} Folgt aus Beispiel 9.3: Sei $\overline{\alpha} \in \kappa_w$ mit $\overline{\alpha}$ separabel.
	Für ein Urbild $\alpha \in \O_w$ ist dann $K(\alpha)/K$  unverzweigt, also $K(\alpha) \subset T$.
	Damit ist auch $\overline{\alpha} \in \kappa_{w_T}$.
\end{proof}

\begin{Bsp}
	Sei $K=\Q_p$ mit $p$-adischer Bewertung $v$ und $\kappa_v = \F_p$.
	Sei $L=\Q_p(\zeta)$, wobei $\zeta$ eine primitive $n$-te Einheitswurzel ist, mit eindeutiger Fortsetzung $w$ von $v$.
	
	\bigskip
	Sei $f\in\O_v[X]$ das Minimalpolynom von $\zeta$. Dann wird $X^n-1$ von $f(X)$ geteilt.
	Sei nun $\overline{f}[X]$ das Bild in $\kappa_v[X] = \F_p[X]$. Dann wird $X^n-1$ auch von $\overline{f}$ geteilt.
	
	\bigskip
	\textbf{(i)} $\ggT(n,p) = 1$:
	
	$X^n-1 \in \kappa_v[X]$ ist separabel, da $NX^[n-1] \neq 0$.
	Also ist $\overline{f}$ separabel und damit folgt nach Beispiel 9.3 und Bemerkung 9.4, dass
	\[ L/ K = \Q_p(\zeta) / \Q_p
	\]
	unverzweigt ist mit $e=1$ und
	\[w \left( \Q_p(\zeta)^\times \right)
	=v\left(\Q_p^\times \right)
	=\Z.
	\]
	
	\bigskip \textbf{(ii)} $n=p^k$:
	
	Da $\zeta$ eine $p^k$-te Einheitswurzel ist, ist $\xi = \zeta^{p^{k-1}}$ eine primitive $p$-te Einheitswurzel. Also gilt
	\[ \xi^{p-1} + \xi^{p-2} + \cdots + 1 = 0
	\]
	so, dass $\zeta$ eine Nullstelle von
	\[ \Phi(X) = X^{(p-1)p^{k-1}} + X^{(p-2)p^{k-1}} + \cdots +1
	\]
	ist. Daher ist $\zeta-1$ eine Nullstelle von $h(x) = \Phi(X+1)$ und $h(x) \in \O_v[X] = \Z_p[X]$ ist nach Eisenstein irreduzibel, denn:
	
	\textbf{(1)} $p$ ist prim in $\O_V = \Z_p$.
	
	\textbf{(2)} $a_0 = h(0) = \Phi(1) = p$ so, dass $p \, | \, a_0$ und $p^2 \not | \,a_0$.
	
	\textbf{(3)} Es gilt
	\begin{align*}
	\Phi(X)
	&= \frac{ \left( X^{p^{k-1}} \right)^p - 1  }{ X^{p^{k-1}} - 1 }
	=\frac{{ X^{p^{k}} - 1 }}{{ X^{p^{k-1}} - 1 }}
	\equiv \frac{{ (X-1)^{p^{k}}}}{{ (X-1)^{p^{k-1}}}}
	\equiv (X-1)^{p^k-p^{k-1}}
	\mod p.
	\end{align*}
	Also ist
	\[ h(X) = \Phi(X+1) \equiv X^{p^k-p^{k-1}} \mod p
	\]
	und somit sind bis auf den führenden Term alle Koeffizienten von h durch $p$ teilbar.
	Es folgt
	\[ [L:K] = p^{k-1}(p-1) = p^k - p^{k-1} = \varphi \left(p^k \right),
	\]
	wobei $\varphi$ die Eulersche $\varphi$-Funktion ist.
	Weiterhin gilt nach der Konstruktion der Bewertungsfortsetzung aus Theorem 5:
	\[ w(\zeta-1)
	= \frac{1}{[L:K]} v \left( \Norm_{L/K} ( \zeta-1) \right)
	= \frac{1}{[L:K]} v(p)
	= \frac{1}{[L:K]}
	\]
	Somit ist $\zeta -1$ ein Erzeuger von $\O_w$ und $L/K$ ist \textbf{rein-verzweigt}, das heißt $f=1$ und $e=[L:K]$. Es gilt also insbesondere $\kappa_w = \kappa_v$.
\end{Bsp}


\begin{Prop}
	Seien $K$ ein lokaler Körper mit Restklassenkörper $\F_q$, $q=p^r$, Bewertung $v$ und $L=K(\zeta)$, $\zeta^n =1$ mit $\ggT(n,p) = 1$. Dann gilt:
	\begin{enumerate}[(i)]
		\item $[L:K]$ ist unverzweigt und somit gilt $[L:K] = [\kappa_w:\kappa_v] =f$.
		\item $[L:K] = f$ ist die kleinste Zahl mit $q^f \equiv 1 \mod n$.
		\item $\Gal(L/K) \cong \Gal(\kappa_w/\kappa_v)$ ist zyklisch und wird von dem Isomorphismus $\zeta \mapsto \zeta^q$ erzeugt.
		\item $\O_w = \O_v[\zeta]$
	\end{enumerate}
\end{Prop}

\begin{proof}
	\enquote{(i)} Wie in Beispiel 9.10.(i).
	
	\bigskip \enquote{(ii)} $X^n-1$ zerfällt über $\O_w$ in Linearfaktoren, da mit $\zeta$ auch alle Potenzen von $\zeta$ in $\O_w$ liegen. Also zerfällt $X^n-1$ auch über $\kappa_v=\F_q$ in Linearfaktoren und damit gilt
	\[ \Z / n\Z \cong U_n = \left\{ 
	\zeta \in \overline{\kappa_w}; \, \zeta^n = 1
	\right\}
	\subset \kappa_w^\times.
	\]
	Dies ist äquivalent zu $n \, | \, q^f-1$. Nach Beispiel 9.3 ist $\kappa_w = \kappa_v(\overline{\zeta})$, also ist $\kappa_w$ die kleinste Körpererweiterung von $\kappa_v$ mit $\kappa_w \supset U_n$. Daraus folgt die Behauptung.
	
	\bigskip \enquote{(iii)} Aus der Algebra folgt, dass $\Gal(\kappa_w/\kappa_v)$ zyklisch ist und von der Frobeniusabbildung  $\overline{\zeta} \mapsto\overline{\zeta}^q$ erzeugt wird.
	Bemerkung 8.8 liefert einen Gruppenhomomorphismus
	\[ \varphi \colon \Gal(L/K) \to \Gal(\kappa_w/\kappa_v).
	\]
	Dieser ist surjektiv, da $\zeta\mapsto \zeta^q$ einen Automorphismus definiert.
	Da $L/K$ unverzweigt ist, gilt
	\[ \# \Gal(L/K) = \#\Gal(\kappa_w/\kappa_v),
	\]
	also ist $\varphi$ ein Isomorphismus.
	
	\bigskip \enquote{(iv)} Da $L/K$ unverzweigt ist, gilt $e=1$ nach Bemerkung 9.2. Somit gilt $\p_w = \p_v \O_w$.
	Außerdem ist $\overline{1}, \overline{\zeta}, \dots, \overline{\zeta}^{f-1}$ eine Basis von $\kappa_w / \kappa_v$ und damit
	\[ \O_w 
	= \O_v[\zeta] + \_w
	= \O_v[\zeta] + \p_v\O_w.
	\]
	Nakayama liefert also $\O_w = \O_v[\zeta]$.
\end{proof}