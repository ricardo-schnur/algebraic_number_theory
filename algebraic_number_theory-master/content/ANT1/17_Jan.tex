\rhead{17 January 2018}

\begin{Bem}
Let $R$ be a DVR with maximal ideal $\m = (\pi)$.
\begin{enumerate}[(i)]
	\item $\pi$ is prime and any prime $\pi'$ is associated to $\pi$.
	\item Any $r \in R \bs \{0\}$ can be written as $r = \varepsilon \pi^k$ with $\varepsilon \in R^\times$ and $k \in\N$ depending only on $r$.
\end{enumerate}
\end{Bem}

\begin{proof}
\enquote{(i)}
Since $R$ is a PID it is also a UFD. Since $\pi$ is prime, if $\pi = ab$ with $a,b \in R$ then
$(\pi) \subset (a)$ such that $(a) = R$ or $(a) = (\pi)$. Hence $a \in R^\times$ or $a = \varepsilon \pi$ with $\varepsilon \in R^\times$. Thus one of $a, b$ must be a unit such that $\pi$ is prime as a irreducible element of a UFD.

\bigskip If $\pi'$ is another prime, then $(\pi') \subset (\pi)$ and we can write $\pi' = a \pi$ with $a \in R$. Since $\pi'$ is prime, $a$ must be a unit and $ \pi' \sim \pi$ follows as claimed.

\bigskip \enquote{(ii)} Since $r$ has a unique prime factorization (up to a unit) the claim follows from (i).
\end{proof}

\begin{Prop}
$R$ is a DVR if and only if $R$ is a local Dedekind domain and not a field.
\end{Prop}

\begin{proof}
\enquote{$\Rightarrow$} $R$ is local since it is a DVR, noetherian since it is a PID and integrally closed since it is a UFD. Furthermore, by Remark 1.8 every prime ideal is maximal. Also, since for the maximal ideal $\m = (\pi)$ we have $\m \neq 0$, $R$ is not a field.

\bigskip \enquote{$\Leftarrow$} $R$ has a unique maximal ideal $\m$ since it is a local ring and $\m \neq 0$ since $R$ is not a field. We need to show that $R$ is a PID:
\begin{enumerate}[(1)]
	\item Show that $\m$ is a principal ideal:
	
			Since $R$ is a Dedekind domain it holds that $\m \neq \m^2$. Let $\pi \in \m \bs \m^2$ and observe that $\m$ is the only non-zero prime ideal.
			Thus, $(\pi) = \m^k$ and $k=1$ since $\pi \not \in \m^2$.
	\item Any ideal $\a$ is a principal ideal since $\a = \m^k = (\pi^k)$.
\end{enumerate}
\end{proof}

\begin{defi}
Let $K$ be a field. A function $v \colon K \to \Z \cup \{\infty\}$ is called \textbf{discrete valuation} if for all $x,y \in K$ the following conditions hold:
\begin{enumerate}[(i)]
	\item $v(xy) =v(x) +v(y)$
	\item $v(x+y) \geq \min \left\{ v(x), v(y) \right\}$
	\item $v(x) = \infty$ if and only if $x = 0$
	\item $v \not\equiv 0$ and $v \not \equiv \infty$
\end{enumerate}
\end{defi}

\begin{Bsp}
Let $p \in \Z$ be prime and $K = \Q$. Define $v_p \colon \Q \to \Z \cup \{\infty\}$ by:
\begin{enumerate}[(1)]
	\item If $z \in \Z \bs \{0\}$ with $z =p^k b$, where $\gcd(p,b) =1$, then $v_p(z) = k$.
	\item If $x \in \Q \bs \{0\}$ with $x = \frac{a}{b}$, where $a,b \in \Z$, then $v_p(x) = v_p(a) -v_p(b)$.
\end{enumerate}
Then $v_p$ is a discrete valuation.
\end{Bsp}


\begin{Prop} 
	\begin{enumerate}[(i)] 
		\item Let $v \colon K \to \Z \cup \{\infty\}$ be a discrete valuation. Then:
				\begin{itemize}
					\item $v(1) = v(-1) = 0$
					\item $v \left( \frac{a}{b} \right) = v(a) -v(b)$
					\item $\O_K = \left\{ x \in K; \, v(x) \geq 0  \right\}$ is a ring with units
							$\O_K^\times = \left\{ x \in K; \, v(x) = 0  \right\}$ 
				\end{itemize}
		\item The ring $\O_K$ from (i) is a DVR.
		\item Conversely, if $R$ is a discrete valuation ring, then there exists a discrete valuation 
				$v \colon K \to \Z \cup \{\infty\}$ with $K= \Quot(R)$ such that $R= \O_K$ for this valuation.
	\end{enumerate}
\end{Prop}

\begin{proof}
\enquote{(i)}
We have 
\[ v(1) = v(1 \cdot 1) = v(1) + v(1)
\]
such that $v(1) = 0$. Furthermore,
\[ 0 = v(1) = v((-1)(-1)) = 2 v(-1)
\]
and hence also $v(-1)=0$. Next,
\[ v(a) = v\left(\frac{a}{b} \cdot b  \right) = v\left(\frac{a}{b} \right) +v(b)
\]
such that $ v\left(\frac{a}{b} \right) = v(a) -v(b)$.
Now, if $v(x) = 0$ then $v\left(\frac{1}{x} \right) =v(1) -v(x) = -v(x) ) =0$ and hence $\frac{1}{x} \in \O_K$, i.e., $x \in \O_K^\times$.
Finally, if $x \in \O_K^\times$ then there is a $y \in \O_K$ with $xy = 1$ such that
\[ 0 = v(1) = v(xy) = \underbrace{v(x)}_{\geq 0} + \underbrace{v(y)}_{\geq 0}
\]
and thus $v(x) = 0$.

\bigskip \enquote{(ii)} 
\begin{itemize}
\item $\O_K$ is an integral domain since $\O_K\subset K$.

\item Show that $\O_K$ is a PID:

Let $\a$ be an ideal in $\O_K$. Choose $a \in \a$ with $v(a)$ minimal. For $b \in \a$ we have $v(b) \geq v(a)$ and hence $v\left(\frac{b}{a} \right) \geq 0$ such that $\frac{b}{a} \in \O_K$.
Thus, $b = \frac{b}{a} \cdot a$ with $\frac{b}{a} \in \O_K$ such that $b \in (a)$ and hence $\a = (a)$.

\item Show that $O_K$ has a unique maximal ideal:

Define $\m = \left\{ a \in \O_K; \, v(a) > 0 \right\}$. Observe that $\m$ is an ideal in $\O_K$ and $\m = \O_K \bs \O_K^\times$ such that $\m$ is a unique maximal ideal in $\O_K$.

\item $\O_K$ is not a field since the valuation $V$ is not allowed to be $0$ everywhere on $K^\times$.
\end{itemize}

\bigskip \enquote{(iii)} 
\begin{enumerate}[(1)]
\item Use Remark 1.8 to define $v \colon R \bs \{0\} \to \N_0, \,  r = \varepsilon \pi^k \mapsto k$.
Observe that $v(ab) =v(a) + v(b)$, $v(a+b) \geq \min\{ v(a), v(b) \}$ and define $v(0) = \infty$.
\item Define $v \colon K = \Quot(R) \to \Z$ by $v\left(\frac{a}{b} \right) = v(a) -v(b)$ if $a \neq 0$ and $v(0) = \infty$ to obtain a discrete valuation $v$.
\item By definition we have $R\subset \O_K$. Show that $\O_K \subset R$:

Let $\frac{a}{b} \neq 0$ be in $\O_K$, i.e., $v\left(\frac{a}{b} \right) \geq 0$. Then we have $v(a) \geq v(b)$.
Let $\m = (\pi) \neq 0$ be the unique maximal ideal of $R$. Then
$a = \varepsilon_1 \pi^{k_1}$ and $b = \varepsilon_2 \pi^{k_2}$ with $\varepsilon_1, \varepsilon_2 \in R^\times$, $k_1,k_2 \in \N_0$ and $k_1 \geq k_2$ such that 
\[ \frac{a}{b} = \frac{\varepsilon_1}{\varepsilon_2} \pi^{k_1-k_2} \in R.
\]
\end{enumerate}
\end{proof}

\begin{Prop}
Let $R$ be an integral domain. Recall that for $\p \in \Spec R$ we have $R \subset R_\p \subset \Quot(R)$. In this situation,
\[ R = \bigcap_{\p\in\Spec R} R_\p.
\]
\end{Prop}

\begin{proof}
	Let $\frac{a}{b} \in \bigcap_{\p\in\Spec R} R_\p$. Consider $\a = \{x \in R; \, xa \in bR \}$ and observe that $\a$ is an ideal. We show that $a \not \subset \p$ for any $\p \in \Spec R$:
	
	Since $\frac{a}{b} \in R_\p$ there are $c \in R, s \in R \bs \p$ with $\frac{a}{b} = \frac{c}{s}$.
	Then we have $as = cb$, which implies $s \in \a$ and $s \not \in \p$.
	
	\bigskip It follows that $\a = R$, i.e., $1 \in \a $ such that $1\cdot a \in bR$ and thus $\frac{a}{b} \in R$.
\end{proof}

\begin{Satz} 
	Let $\O$ be a noetherian integral domain. Then $\O$ is a Dedekind domain if and only if $\O_\p$ is a DVR for all $\p \in \Spec(\O) \bs \{ 0 \}$.
\end{Satz}

\begin{proof}
\enquote{$\Rightarrow$} 
By Proposition 1.6, $\O_\p$ is a Dedekind domain and by Proposition 1.5, $\O_\p$ is local.
Since $\p \neq 0$ we have $\p S^{-1} \neq 0$\todo{hier ist was falsch} and hence $\O_p$ is not a field. Hence, $\O_\p$ is a DVR by Proposition 1.9.

\bigskip \enquote{$\Leftarrow$}
By Proposition 1.11 we have $\O = \bigcap_{\p\in\Spec \O} \O_\p$. Furthermore, Proposition 1.9 implies that $\O_\p$ is integrally closed for any $\p \in \Spec(\O)$ and hence the same holds true for $\O$.

\bigskip
Show: Every prime ideal $\p \neq 0$ in $\O$ is maximal.

Consider $\p \subset \m$ for some maximal ideal $\m$. Consider the localisation $\O_\m = \O S^{-1}$ with $S = \O \bs \m$. Then we obtain $\p S^{-1} \subset \m S^{-1}$ and $\p S^{-1}, \m S^{-1}$ both are prime ideals by Proposition 1.3.
Since $\O_\m$ is a DVR, Proposition 1.9 implies that $\O_\m$ is a Dedekind domain, whence
$\p S^{-1} = \m S^{-1}$, such that by Proposition 1.3 we may finally conclude $\p = \m$.
\end{proof}














