\rhead{20 November 2017}

Let $\Gamma = v_1 \Z + \dots+ v_n \Z \subset \R^n$ be a complete lattice. We define
\[ \vol \Gamma = \vol \phi = \abs{ \det (v_1, \dots, v_n)}.
\]
Note that this definition is independent of the chosen basis since for a transformation
\[ A (v_1, \dots, v_n)=(v_1', \dots, v_n')
\]
between two bases we have $\det A = \pm 1$.

\begin{Satz}[Minkowski]
Let $X \subset \R^n$ be a convex, symmetric central (i.e., $x \in X$ implies $-x \in X$) subset and let $\Gamma \subset \R^n$ be a complete lattice. If
\[ \vol X> 2^n \vol \Gamma
\]
then there exists some $\gamma \in \Gamma\bs \{0\}$ such that $\gamma \in X$.
\end{Satz}

\begin{proof}
\textbf{Claim:} It suffices to show that there are $\gamma_1, \gamma_2 \in \Gamma$, $\gamma_1 \neq \gamma_2$, such that
\[ \left( \frac{1}{2}X + \gamma_1 \right) \cap \left( \frac{1}{2}X + \gamma_2 \right)
\neq \emptyset.
\]

\textbf{Proof of claim:} Let $x = \frac{1}{2} x_1 + \gamma_1 = \frac{1}{2} x_2 + \gamma_2$ with some $x_1, x_2 \in X$. Then
\[ y= \frac{1}{2} \left( x_1-x_2 \right) = \gamma_2- \gamma_1 \in \Gamma \bs \{ 0 \}
\]
with $y \in X$ since $X$ is symmetrical central.

\bigskip
Now let us assume that the family $\left( \frac{1}{2}X + \gamma \right)_{ \gamma \in \Gamma }$ is pairwise disjoint. Then
\[ \left(\left[ \frac{1}{2}X + \gamma \right] \cap \phi \right)_{ \gamma \in \Gamma }
\]
also consists of pairwise disjoint sets such that we obtain the contradiction
\begin{align*}
\vol \Gamma
& = \vol \phi
\geq \sum_{\gamma \in \Gamma  } \vol \left(   \left[ \frac{1}{2}X + \gamma \right] \cap \phi\right)
= \sum_{\gamma \in \Gamma  } \vol \left(  \frac{1}{2}X \cap \left[  \phi - \gamma \right] \right) \\
& = \vol \left( \frac{1}{2}X \right)
= \frac{1}{2^n} \vol X.
\end{align*}
\end{proof}




\section{Minkowski theory}
Let $[K:\Q] = n$ be a field extension, $\tau_i \colon K \hookrightarrow \C$ different embeddings and consider the embedding
\[ j \colon K \hookrightarrow K_\C = \prod_{\tau_i} \C, \,
a \mapsto \left( \tau_1(a), \dots, \tau_n(a)  \right).
\]
Define a hermitian scalar product on $K_\C$ by
\[ \langle \left(x_{\tau_i} \right), \left(y_{\tau_i} \right) \rangle 
=\sum_{\tau_i} x_{\tau_i} \overline{y_{\tau_i}}
\]
and consider the complex conjugation $F \in \Gal(\C/ \R)$ given by $\C \to \C, z \mapsto \overline{z}$. Let
\[ F(\tau) = \overline{\tau} \colon a \mapsto \overline{\tau(a)}
\]
and extend it to $K_\C$ by
\[ F \colon K_\C \to K_\C, \, \left(x_{\tau} \right) 
\mapsto \left(\overline{x}_{\overline{\tau}} \right).
\]

\begin{Bsp*}
Let $D > 0$ be square-free. Consider
\[ \Q\left( \sqrt{D} \right) \hookrightarrow \Q\left( \sqrt{D} \right)_\C
= \prod_{\tau_i} \C
\]
with
\[ \tau_1\left( a+b\sqrt{D} \right) = a+b\sqrt{D} 
\qquad \text{and} \qquad
\tau_2\left( a+b\sqrt{D} \right) = a-b\sqrt{D}.
\]
Then
\[ j \left( a+b\sqrt{D} \right) = \left( a+b\sqrt{D} ,  a-b\sqrt{D}\right)
\]
and $F(\tau_1) = \tau_1, F(\tau_2) = \tau_2$ such that
\[ F\left( x_{\tau_1}, x_{\tau_1}  \right) 
= \left( \overline{x}_{\tau_1}, \overline{x}_{\tau_2}  \right).
\]
\end{Bsp*}

\bigskip
\begin{Bem*}\begin{itemize}
\item $F(\langle x, y \rangle ) = \langle F(x), F(y) \rangle$
\item $\Tr \colon K_\C \to \C, \, \left( x_\tau\right) \mapsto \sum_\tau x_\tau$
such that $(\Tr \circ j) (a) = \Tr_{K/\Q}(a)$
\end{itemize}
\end{Bem*}


\bigskip
Now define the $F$-invariant $\R$-vector space
\[ K_\R = K_\C^+ 
=\{ x \in K_\C\, | \, F(x) = x \}
=\{ x \in K_\C\, | \, x_{\overline{\tau}} = \overline{x_\tau} \text{ for all } \tau \}.
\]
Since $\overline{\tau}(a) = \overline{\tau(a)}$ for all $a \in K$ and all $\tau$, we have $j(K) \subset K_\R$. We call $K_\R$ the \textbf{Minkowski space} and
$\langle \cdot, \cdot \rangle \big|_{K_\R}$ the \textbf{canonical metric}.


\bigskip
\begin{Bem*}
Note that $j \colon K \to K_\R \cong K \otimes_\Q \R$, where the isomorphism is given by $a \otimes x \mapsto j(a) x$ for $x \in \R$.
\end{Bem*}


\bigskip
\textbf{Explicit description of $K_\R$:} Let $n = r+2s$, where $r$ and $s$ are the number of embeddings
\[ \varphi_1, \dots, \varphi_r \colon K \hookrightarrow \R
\]
and
\[ \sigma_1, \overline{\sigma_1}, \dots, \sigma_s, \overline{\sigma_s} \colon K \hookrightarrow \C,
\]
respectively. Notice that $F(\varphi_i) = \varphi_i$ and $F(\sigma_j) = \overline{\sigma_j}$.
Then elements of $K_\C$ are of the form
\[ x = \left( x_{\varphi(1)}, \dots, x_{\varphi(r)},  x_{\sigma_1}, x_{\overline{\sigma_1}}, \dots, x_{\sigma_s}, x_{\overline{\sigma_s}} \right)
\]
with
\[ F(x) = \left( \overline{x_{\varphi_1}}, \dots, \overline{x_{\varphi_r}},  \overline{x_{\overline{\sigma_1}}}, \overline{x_{\sigma_1}}, \dots, \overline{x_{\overline{\sigma_s}}}, \overline{x_{\sigma_s}} \right).
\]
Hence we have
\[ K_\R = \left\{
x \in K_\C \, \big| \, x_{\varphi_i} \in \R, x_{\overline{\sigma_j}} = \overline{x_{\sigma_{j}}}
\right\}.
\]


\begin{Prop}
The map
\begin{align*}
f \colon K_\R &\xrightarrow{\cong} \R^{r+2s} = \prod_\tau \R, \\
x & \mapsto \left( x_{\varphi_1}, \dots, x_{\varphi_r}, \Re x_{\sigma_1} , \Im x_{\sigma_1}, \dots, \Re x_{\sigma_s} , \Im x_{\sigma_s}.
\right)
\end{align*}
is an isomorphism. It transforms the canonical metric into the scalar product
\[ \langle x, y \rangle = \sum_\tau \alpha_\tau x_\tau y_\tau,
\]
where \[
\alpha_\tau = \begin{cases}
1, & \tau = \varphi_i \text{ for some } i, \\
2, & \tau = \sigma_j \text{ for some } j.
\end{cases}
\]
\end{Prop}

\begin{proof}
Obviously, $f$ is an isomorphism. For $x=(x_\tau), y =(y_\tau) \in K_\R$ we have
\begin{align*}
\langle x, y \rangle \big|_{K_\R}
&= \sum_\tau x_\tau \overline{y_\tau} \\
&= \sum_{\varphi_i} x_{\varphi_i} y_{\varphi_i}
	+\sum_{\sigma_j} x_{\sigma_j} \overline{y_{\sigma_j}}
	+\sum_{\overline{\sigma_j}} \overline{\left(x_{\sigma_j} \overline{y_{\sigma_j}}\right)} \\
&= \cdots = \left( f(x), f(y) \right).
\end{align*}
\end{proof}

\begin{Bem*}
\begin{itemize}
\item The canonical metric induces a volume $\volcan$ on $K_\R$ and thus on $\R^{r+2s}$.
\item If we denote the Lebesgue measure on $\R^{r+2s}$ by $\volLeb$  then, for $X \subset K_\R$,
\[ 2^{s} \volLeb f(X) = \volcan X.
\]
\item We will thus consider $ K \supset U \xrightarrow{j} j(U) \xrightarrow{f} \R^{r+2s}$.
\end{itemize}
\end{Bem*}


















