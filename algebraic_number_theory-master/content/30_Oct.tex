\rhead{30 October 2017}


Recall: $L=\Q(\zeta)$, $\O = \Z[\zeta]$, where $\zeta$ is a $p$-th root of unity

\bigskip
\textbf{Last time:}
\begin{enumerate}[(1)]
	\item $a_1 1 +a_2 \zeta + \cdots + a_p\zeta^{p-1} =\alpha$ and at least one $a_j=0$
	
	If $\alpha$ is divided by $n \in \Z$ then all the $a_i$ are divided by $n$.
	

\item $x+y\zeta-x\zeta^{2s} -y\zeta^{2s-1} \equiv 0 \mod p$
\end{enumerate}

\begin{proof}[Continuation of proof of Theorem 1]
	\enquote{Case 2} $1, \zeta, \dots, \zeta^ {2s}$ are not distinct.
	
	Observe: $1 \neq \zeta$ and $\zeta^ {2s-1} \neq \zeta^ {2s}$
	
	\bigskip \enquote{Case 2A} $1 = \zeta^ {2s} (\Leftrightarrow p|s)$.
	
	(2) implies $y\zeta -y\zeta ^{2s-1} \equiv 0 \mod p$ such that Remark 2.12 yields the contradiction $p|y$.
	
	\bigskip \enquote{Case 2B}  $1 = \zeta^ {2s-1} (\Leftrightarrow \zeta = \zeta^ {2s})$.
	
	(2) implies $(x-y)1 + (y-x) \zeta \equiv 0 \mod p$ such that Remark 2.12 yields $p | y-x$, which contradicts the assumption $x \not \equiv y \mod p$.
	
	\bigskip \enquote{Case 2C} $\zeta = \zeta^ {2s-1}$.
	
	(2) implies $x-x\zeta^2 \equiv 0 \mod p$ such that Remark 2.12 yields the contradiction $p|x$.
\end{proof}

\textbf{Questions:}
\begin{enumerate}[(1)]
\item Under which assumption is $\O$ a UFD?
\item What can we do if $\O$ is not a UFD?

$\rightarrow$ Idea of Kummer: \enquote{calculate with ideals}
\end{enumerate}

\bigskip\textbf{Prospect:} Theorem (Montgomery, Uchida, 1971)

$\Z[\zeta]$ is a UFD if and only if $p\leq 19$, $p$ prime.

\bigskip
\textbf{Preview:} From Kummer's idea we obtain a better criterion
 for $p$ called \textbf{regular}, which ensures that Fermat's conjecture holds for $p$.
 

 \begin{conjecture*}There are infinitely many regular primes.
 \end{conjecture*}





\chapter{Ring of integers}

In this chapter, all rings are assumed to be commutative with $1$.


\section{Integral ring extensions}

\begin{defi}[\enquote{ganze Ringerweiterungen}]
	Let $A\subset B$ be a ring extension.
\begin{enumerate}[(i)]
\item $b \in B$ is \textbf{integral } over $A$ if there exists a monic polynomial
$f(X)=X^n+a_{n-1}X^{n-1}+\dots + a_0 \in A[X]$ with $f(b)=0$.
\item $B$ is \textbf{integral } over $A$ if all $b\in B$ are integral over $A$.
\end{enumerate}
\end{defi}

\begin{Prop}
	Let $A\subset B$ be a ring extension and $b_1,\dots,b_n \in B$. Then
	$b_1,\dots, b_n$ are integral over $A$ if and only if 
	\[ A[b_1,\dots, b_n] = \{ f(b_1,\dots,b_n) \, | \, f \in A[X_1,\dots, X_n]
	\}
	\]
	is a finitely generated $A$-module.
\end{Prop}

\begin{remin}[\enquote{Adjunkte}]
Let $R$ be a ring and $A \in R^{n \times n}$
\begin{enumerate}[(i)]
\item $A^\# = (a_{i,j}^\#)$ with $a_{i,j}^\# = (-1)^{i+j} \det(A_{j,i})$,
	where $A_{j,i}$ is obtained from $A$ by deleting the $j$-th row and $i$-th column of $A$.
\item We have $AA^\# = A^\#A = \det(A) I$.
In particular, $Ax=0$ implies $A^\#Ax=0$ such that $\det(A)x = 0$.
\end{enumerate}
\end{remin}

\begin{proof}[Proof of Proposition 1.2]
\enquote{$\Rightarrow$} If $n=1$ and $b$ is integral over $A$, then there is an $f \in A[X]$ with $f$ monic such that $f(b)=0$. Let $g \in A[X]$ be arbitrary. Then 
\[ g(X) = q(X)f(X)+r(X)
\]
with $q,r \in A[X]$ and $\deg r < \deg f = d$. Hence $g(b)=r(b)$ with $\deg r < d$. Thus $\{1,b, \dots, b^{d-1} \}$ generate $A[b]$ as an $A$-module.
The case $n\geq 2$ follows by induction.

\bigskip \enquote{$\Leftarrow$}
$A[b_1,\dots,b_n]$ is finitely generated as an $A$-module by $w_1, \dots, w_r$.
If $b \in A[b_1,\dots,b_n]$ then
\[ bw_i = \sum_{j=1}^{r} a_{j,i}w_j
\]
such that
\[ \left( bI - (a_{i,j}) \right)w = 0.
\]
Thus, $\det \left( bI - (a_{i,j}) \right)w = 0$ and hence 
\[\det \left( bI - (a_{i,j}) \right)w_i = 0
\]
for all $i=1,\dots, r$. If we now use that 
\[ 1 = c_1w_1+ \cdots + c_rw_r
\]
we can infer $\det \left( bI - (a_{i,j}) \right)1 = 0$. Consider
\[M= bI - (a_{i,j}) = \begin{pmatrix}
b-a_{1,1} & -a_{1,2} & \cdots & -a_{1,r} \\
-a_{2,1} & b-a_{2,2} & \cdots & -a_{2,r} \\
\vdots & \vdots & \ddots & \vdots \\
-a_{r,1} & -a_{r,2} & \cdots & b-a_{r,r} \\
\end{pmatrix}.
\]
By the Leibniz formula we have
\[ \det(M)= \sum_{\sigma\in S_n} \sgn(\sigma ) \prod_{i=1}^n m_{\sigma(i),j}
\]
which is a polynomial over $b$ with leading coefficient $1$. Hence $b$ is integral over $A$.
\end{proof}


\begin{Kor}[And Definition]
\begin{enumerate}[(i)]
	\item If $A \subset B$ is an extension of rings then
	\[ \overline{A} = \{ b \in B \, | \, b \text{ is integral over } A
	\}
	\]
	is a ring. It is called the \textbf{integral closure } of $A$ in $B$.
	If $\overline{A} = A$ then $A$ is called \textbf{integrally closed}  in $B$.
	\item We have transitivity, that is to say, if $A,B,C$ are rings with $A \subset B \subset C$ such that $C$ is integral over $B$ and $B$ is integral over $A$ then $C$ is integral over $A$.
	\item The integral closure of $A$ in $B$ is integrally closed, i.e., 
	$\overline{\overline{A}} = \overline{A}$.
\end{enumerate}
\end{Kor}


\begin{proof}[Proof]
\enquote{(i)} If $b_1, b_2 \in \overline{A}$ then $A[b_1], A[b_2]$ are finitely generated $A$-modules. Hence $A[b_1,b_2]$ is a finitely generated $A$-module.
Thus, by Proposition 1.3, $b_1+b_2$ and $b_1b_2$ are integral, i.e., elements of $\overline{A}$.

\bigskip \enquote{(ii)} If $c\in C$ then $c$ is integral over $B$ and hence there is a monic polynomial $f= X^n + b_{n-1}X^{n-1} + \dots + b_0 \in B[X]$ with $f(b)=0$. This shows that $c$ is integral over $R=A[b_1,\dots,b_{n-1}]$ such that Proposition 1.3 shows that $R[c]$ is a finitely generated $R$-module.
Furthermore, $b_0,\dots, b_{n-1}$ are integral over $A$ such that another application of Proposition 1.3 shows that $R$ is a finitely generated $A$-module.
Hence, $R[c]$ is a finitely generated $A$ module such that $c$ is integral over $A$ by Proposition 1.3.

\bigskip \enquote{(iii)} Follows from (ii).
\end{proof}

\begin{defi}[\enquote{ganzer Abschluss und normaler Ring}]
If $A$ is an integral domain we call its integral closure $\overline{A}$ in $K = \Quot(A)$ the \textbf{normalization} or the \textbf{integral closure} of $A$. We say $A$ is \textbf{integrally closed} if $A$ is integrally closed in $K$.
\end{defi}


\begin{Bem}
	If $A$ is a UFD then $A$ is integrally closed.
\end{Bem}

\begin{proof}
Suppose $b = \frac{a}{a'} \in \Quot(A)$ with $\gcd(a,a') =1$ is integral over $A$.
Then there exist $a_0, \dots, a_{n-1} \in A$ with
\[ \left( \frac{a}{a'} \right)^n + a_{n-1} \left(\frac{a}{a'} \right)^{n-1}
+ a_{n-2} \left(\frac{a}{a'} \right)^{n-2} + \dots + a_0 = 0
\]
such that
\[ a^n+a_{n-1}a'a^{n-1} + a_{n-2}a'^{\, 2} a^{n-2} + \dots + a_0 a'^{\, n} = 0.
\]
Let $a' = \varepsilon \pi_1 \cdots \pi_r$ be the prime factorization of $a'$ with $\varepsilon \in A^\times$ and $\pi_1, \dots, \pi_r$ primes.
Since $\pi_i | a'$ the above equation shows that actually $\pi_i | a^n$. But this implies $\pi_i | a$ which is a contradiction to $\gcd(a,a') = 1$.
Hence we have $a' = \varepsilon \in A^\times$ such that $b \in A$.
\end{proof}



\section{Integral closures in field extensions}

\textbf{Setting:}
\begin{itemize}
\item $A$ is an integral domain.
\item $A$ is integrally closed.
\item $K= \Quot(A)$.
\item $L / K$ is a finite field extension with $\overline{A}_K = A \subset K = \Quot(A) \hookrightarrow L \supset B = \overline{A}_L$.
\item $B$ is the integral closure of $A$ in $L$. Observe: $B \cap K = A$
\end{itemize}

\begin{Bem}
\begin{enumerate}[(i)]
	\item $B$ is integrally closed in $L$.
	\item If $\beta \in L$ then there are $b \in B$ and $a \in A\bs \{0\}$ such that $\beta = \frac{b}{a}$. 
	
	In particular, $L=\Quot(B)$.
	\item For $\beta\in L$ we have $\beta \in B$ if and only if $f_\beta \in A[X]$, where $f_\beta$ is the minimal polynomial of $\beta$ over $K$.
\end{enumerate}
\end{Bem}

\begin{proof}\enquote{(i)} Follows from the transitivity in Corollary 1.4.
	
	\bigskip \enquote{(ii)} Choose $a\in A$ with $a^n f_\beta(X) = a^n X^n+a^{n-1}c_{n-1}X^{n-1} + \dots + c_0 \in A[X]$.
	Then we have
	\[ a^n \beta^n+c_{n-1}a^{n-1}\beta^{n-1} + \dots + c_0=0
	\]
	and hence
	\[ (a \beta)^n+c_{n-1}(a\beta)^{n-1} + \dots + c_0=0
	\]
	such that $a\beta$ is integral over $A$. Consequently, $b=a \beta \in B$ and $\beta = \frac{b}{a}$.
	
	\bigskip \enquote{(iii)} \enquote{$\Leftarrow$} Obvious. \enquote{$\Rightarrow$} Let $\beta$ be a zero of 
	$g(X)=X^n+a_{n-1}X^{n-1}+\dots + a_0 \in A[X]$. Then $f_\beta$ divides $g$.
	If $\beta_1,\dots, \beta_n$ are the zeros of $f_\beta$ in $\overline{K}$ then they are also zeros of $g$ and thus integral over $A$. Hence the coefficients of $f_\beta$ are integral over $A$ and are elements of $K$ such that $f_\beta \in A[X]$ as claimed.
\end{proof}