\rhead{07 Mai 2018}

\begin{Kor}
In Satz 3 (Satz von Ostrawski) erhalte sogar Isomorphismus von normierten Körpern. Also $(K, |\cdot|)$ vollständiger Körper mit archimedischer Norm, dann:
\[ (K,|\cdot|) \cong (\R, |\cdot|_\infty) \text{ oder } (\C, |\cdot|_\infty)\]
\end{Kor}

\begin{proof}
Sei $(K,|\cdot|)$ ein solcher Körper. Im Beweis von Satz 3: $\R \hookrightarrow K$ mit Einbettung erhält die Norm. $K/ \R$ ist $\leq$ Grad 2 Erweiterung.\\
Prop 6.8+ Prop 1.7 $\Rightarrow$ Behauptung.
\end{proof}

\begin{Lem}
Sei $K$ ein Körper und $h: K \to \R_{\geq 0}$ Abbildung mit
\[h(\alpha) = 0 \iff \alpha = 0 \text{ und } \forall \ \alpha,\beta: h(\alpha \beta) = h(\alpha)h(\beta)\]
Dann sind äquivalent:
\begin{itemize}
\item $h$ erfüllt die verschärfte Dreiecksungleichung $\forall\ \alpha, \beta: h(\alpha+\beta) \leq \max\{h(\alpha), h(\beta)\}$
\item Es gilt: $\forall \alpha \in K: h(\alpha) \leq 1 \Rightarrow h(\alpha+1) \leq 1$
\end{itemize}
\end{Lem}

\begin{proof}
\glqq $\Rightarrow$\grqq : mit $\beta=1$.\\
\glqq $\Leftarrow$ \grqq : $\alpha=\beta=0$: \checkmark\\
$\OE h(\alpha) \leq h(\beta)$ und $\beta \neq 0 \Rightarrow h(\frac{\alpha}{\beta})=\frac{h(\alpha)}{h(\beta)} \leq 1 \Rightarrow h(\alpha+\beta)=h(\beta)h(\frac{\alpha}{\beta}+1)\stackrel{Vor.}{\leq} h(\beta)\cdot 1 ) \max \{h(\alpha), h(\beta)\}.$
\end{proof}

\begin{proof}[Beweis von Satz 5]
\begin{enumerate}[(A)]
\item Falls $||\cdot||$ archimedisch, dann folgt Beh. aus:
\begin{itemize}
\item $\C$ ist algebraischer Abschluss von $\R$
\item Kor. 6.9
\item $|z|_\infty = \sqrt{z \bar{z}} = \sqrt[2]{|\mathcal{N}_{\C/\R}(z)|}$ für $z \in \C$.
\end{itemize}
\item Sei nun $||\cdot||$ nicht-archimedisch und $L/K$ endlich. Zeige, dass $\alpha \mapsto \sqrt[n]{||\NLK(\alpha)||}=:h(\alpha)$ Norm ist.
\begin{enumerate}[(1)]
\item $h(\alpha)=0=\NLK(\alpha)=0 \iff \alpha=0$
\item $h(\alpha\beta)=h(\alpha)h(\beta)$
\item Seien $\O=\{\alpha \in K \ | \ ||\alpha|| \leq 1 \}$ der BWR zu $||\cdot||$, und $\hO$ der ganze Abschluss von $\O$ in $L$.\\
Lemma 6.6. $\Rightarrow$ $\hO=\{\alpha \in L \ | \ \NLK(\alpha) \in \O\} = \{\alpha \in L \ | \ h(\alpha) \leq 1\}$.\\
Lemma 6.10. $\Rightarrow$ die verschärfte Dreiecksungleichung gilt.\\
Prop. 6.8. $\Rightarrow$ Jede Fortsetzung der Norm auf $K$ nach $L$ ist äquivalent zu $h \Rightarrow$ Eindeutigkeit + Vollständigkeit.
\end{enumerate}
\item Sei nun $|| \cdot||$ nicht-archimedisch und $L/K$ algebraisch. Für $\alpha \in L$ wähle endlichen Zwischenkörper $E$ von $L/K$. Definiere $|\alpha|:=\sqrt[n]{||\mathcal{N}_{E/K}(\alpha)||}$, wobei $n:=[E:K]$.\\
Dies ist wohldefiniert, da für $\alpha \in F=E_1 \cap E_2$ mit $n_i:=[E_i : F], d:=[F:K]$ gilt:
\[\sqrt[n_1\cdot d]{|| \mathcal{N}_{E_1/K} (\alpha)||}=\sqrt[n_1\cdot d]{|| \mathcal{N}_{F/K}\underbrace{(\mathcal{N}_{E_1/F} (\alpha))}_{=\alpha^{n_1}}||}=\sqrt[d]{||\mathcal{N}_{F/K}(\alpha)||}=\sqrt[n_2\cdot d]{|| \mathcal{N}_{E_2/K}}\]
Außerdem ist das eine Norm, da die Bedingungen auf allen endlichen Zwischenkörpern erfüllt ist.
\end{enumerate}
\end{proof}

\begin{Kor}
In der Situation von Satz 5 gilt für nicht-archimedische Körper $K$ und endliche KE $L/K$:
\begin{enumerate}[(i)]
\item Die Bewertung auf $K$ ist diskret $\iff$ die Bewertung auf $L$ ist diskret.
\item $\hO:= \text{ der ganze Abschluss von }\O \text{ in }L \\
\hphantom{\hO \,}= \{x \in L \ | \ \NLK(x) \in \O\}\\
\hphantom{\hO \,}= \{x \in L \ | \ ||x|| \leq 1 \} = \{x \in L \ | \ \hat{\nu}(x) \geq 0\}$\\
\[\begin{tikzcd}
\p\hO=\hat{\p}^e \arrow[draw=none]{r}[sloped,auto=false]{\subset} & \hO \arrow[draw=none]{r}[sloped,auto=false]{\subset} &L\\ %\arrow[r, subset]
\p \arrow[draw=none]{r}[sloped,auto=false]{\subset}  & \O \arrow[draw=none]{r}[sloped,auto=false]{\subset} \arrow[hook]{u}{\text{ganzer Abschluss}} &K \arrow[hook, swap]{u}{endlich}
\end{tikzcd}\]

Wir sind also im Setting von Algebraischer Zahlentheorie I:\\
$\p$ ist eindeutiges Primideal $\neq 0$ in $\O$. Dieses hat ein \glqq Urbild\grqq \ $\hp$ in $\hO$ und es gilt also $\p\hO=\hp^e$.\\
$\p$ ist also voll-verzweigt und der Trägheitsgrad bzw. lokale Grad ist $[\hat \kappa:\kappa]$ mit $\hat \kappa:=\hO/\hp, \kappa:=\O/\p$.
\end{enumerate}
\end{Kor}

\begin{Bem}
Seien $L/K$ Körpererweiterung von normierten Körpern mit nicht-archimedischer Norm.\\
Notation: $\hat{v}, \hO, \hp, \hat{\kappa}$ zu $L$ wie immer, $v, \O, \p, \kappa$ zu $K$ wie immer.\\
$L/K$ endlich $\Rightarrow [\hat{\kappa}: \kappa]$ endlich.
\end{Bem}

\begin{proof}
Seien $\bar{x_1}, \dots, \bar{x_n} \in \hat{\kappa}= \hO/\hP$ linear unabhängig über $\kappa$.
Wähle Urbilder $x_1, \dots, x_n \in \hO \subseteq L$ und zeige: diese sind linear unabhängig über $K$.\\
Seien $c_1, \dots, c_n \in K$ mit $c_1 x_1 + \dots + c_nx_n=0$ nicht alle $0$.\\
Durchmultiplizieren mit geeignetem $c\in K \Rightarrow \OE c_1, \dots, c_n \in \O$ und ein $c_i \not \in \p$\\
$\Rightarrow \bar{c_1}\bar{x_1}+\dots+\bar{c_n}\bar{x_n}=0$ mit $\bar{c_i} \neq 0$ in $\hat{\kappa}$.
\end{proof}

\section{Lokale Körper}
\begin{defi}
\begin{enumerate}[i)]
\item Ein Körper $K$ heißt \textbf{\underline{globaler Körper}} $:\iff$\\
$K$ ist Zahlkörper, d.h. endliche Erweterung von $\Q$, oder\\
$K$ ist endliche Erweterung von $\F_q(T)=\Quot(\F_q[T])$ mit $\F_q=$ Körper mit $q$ Elementen.
\item Ein Körper $K$ heißt \textbf{\underline{lokaler Körper}} $: \iff$\\
$K$ ist bezüglich der Norm von einer nicht-archimedischen Bewertung $v$ vollständig, $v$ ist diskret und der Restklassenkörper $\kappa$ ist endlich.
\end{enumerate}
\underline{Notation:} Für $K$, Bwertung $v$, $\O$, $|\cdot|=r^{-v(\cdot)}$ mit $r >1, \p, \kappa=\O/\p$
\end{defi}

\begin{Prop}
Sei $K$ lokaler Körper. Betrachte die von $|\cdot|$ induzierte Topologie auf $K$. Dann ist $\O$ kompakt und $K$ lokal kompakt.
\end{Prop}

\begin{Err}
Sei $(X,T)$ ein topologischer Raum der hausdorffsch ist. Dann ist $(X,T)$ lokal-komtakt $\iff$\\
$\forall \ x \in X \exists$ Umgebung $C$ von $x$, die kompakt ist, d.h.\\
$\forall \ x \in X \exists \ \U \in T, C$ kompakt mit $x \in U \subseteq C$.\\
\danger Für nicht hausdorffsche Räume gibt es unterschiedliche Definitionen. %$\hat{\underline{/!\bs}}$
\end{Err}

\begin{proof}
Erinnerung: Für die Kette $ \O \supset \p \supset \p^2 \supset \p^3 \supset \dots$ gilt:\\
$\p^k/\p^{k+1} \cong \O/\p = \kappa$ endlich.\\
Aus $\O/\p$ endlich folgt, dass $\O/\p^n$ endlich für alle $n \in \N$.\\
$\Rightarrow \prod_{n \in \N} \O/p^n$ ist kompakt (Satz von Tychonoff).\\
$\O \cong \lim_{\leftarrow n} \O/\p$ ist abgeschlossen in $\prod_n \O/\p^n$ und damit ebenfalls kompakt.\\
Außerdem: $\O$ ist offen in $K$, denn:\\
$x \in \O \iff |x| \leq 1 \iff v(x) \geq 0 \stackrel{\OE v(K)=\Z\cup \{\infty\}}{\iff} v(X) >-1 \iff |x|<r^{-1} \Rightarrow \O$ ist offene Kreisscheibe um 0.\\
$\Rightarrow a \in K$ ist also $a+\O$ eine offene und kompakte Umgebung und somit $K$ lokal kompakt.
\end{proof}