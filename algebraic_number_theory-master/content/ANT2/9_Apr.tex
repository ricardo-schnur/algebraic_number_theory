\rhead{09 April 2018}

\chapter{Bewertungstheorie}
\section{Normierte Körper}

\begin{defi}
Eine \textbf{Norm} auf einem Körper $K$ ist eine Funktion
$\abs{\cdot} \colon K \to \R_{\geq 0}$ mit:
\begin{enumerate}[(i)]
\item $\abs{x} \geq 0$ und $\abs{x} = 0 \Leftrightarrow x = 0$
\item $\abs{xy} = \abs{x} \abs{y}$
\item $\abs{x+y} \leq \abs{x} + \abs{y}$
\end{enumerate}
\end{defi}

\begin{Bem}
\begin{enumerate}[(i)]
\item $\abs{1} = \abs{-1} = 1$ ud $\abs{-x} = \abs{x}$
\item $\abs{\cdot} \colon K \to \R_{\geq 0}$ mit $\abs{0} = 0$ und $\abs{x} = 1$ für $x\neq 0$ is eine Norm und heißt \textbf{triviale Norm}.
\end{enumerate}
\end{Bem}

Ab jetzt: $K$ immmer ein Körper

\begin{Bem}
Jede Norm definiert eine Metrik auf $K$ durch $d(x,y) = \abs{x-y}$.
\end{Bem}

\begin{Bem}
Jede Metrik auf einer Menge $X$ definiert eine Topologie auf $X$ durch:
\[ U \subset X \text{ offen } \Leftrightarrow \text{ für alle }  x \in U \text{ existiert } \varepsilon > 0
\text{ mit } K_\varepsilon(x) \subset U \]
Hier: $K_\varepsilon(x) = \left\{ y \in X \, | \, d(x,y) < \varepsilon  \right\}$
\end{Bem}

\begin{recall*}
Für eine Menge $X$ heißt $T \subset \Pot(X)$ \textbf{Topologie} falls
\begin{enumerate}[(i)]
\item $\emptyset, X \in T$,
\item $U, V \in T$ impliziert $U \cap V \in T$,
\item $U_i \in T$ für alle $i \in I$ impliziert $\bigcup_{i \in I} U_i \in T$.
\end{enumerate}
\end{recall*}


\begin{Bem}
Seien $X$ eine Menge, $d$ eine Metrik auf $X$ und $T$ die induzierte Topologie. Dann hängt Konvergenz von Punktfolgen nur von der Topologie ab.

\bigskip
Genauer: Für eine Folge von Punkten $x_n \in X$ und $x \in X$ gilt:

$x_n \xrightarrow{n \to \infty} x \Leftrightarrow$  Für alle $U \in T$ mit $x \in U$ ein $N \in \N$ existiert mit $x_n \in U$ für alle $n \geq N$.
\end{Bem}

\begin{proof}
\enquote{$\Rightarrow$} Zu $U \in T$ existiert $ r>0$ mit $K_r(x) \subset U$. Wähle $N \in \N$ mit $d(x_n,x) < r$ für alle $n\geq N$. Dann ist $x_n \in K_r(x) \subset U$ für alle $N \geq N$.

\bigskip \enquote{$\Leftarrow$} Sei $\varepsilon > 0$. Wähle $U = K_\varepsilon(x)$. Da $U$ offen ist, existiert $N \in \N$ mit $x_n \in U$ für alle $n\geq N$. Also ist $d(x_n,x) < \varepsilon$ für alle $N\geq N$.
\end{proof}


\begin{defi}
Zwei Normen $\abs{\cdot}_1$ und $\abs{\cdot}_2$ auf $K$ heißen \textbf{äquivalent}, $\abs{\cdot}_1 \sim \abs{\cdot}_2$, wenn sie die selbe Topologie auf $K$ induzieren.
\end{defi}


\begin{Prop}
$\abs{\cdot}_1$ und $\abs{\cdot}_2$ sind äquivalent genau dann, wenn ein $s > 0$ existiert mit $\abs{\cdot}_2 = \abs{\cdot}_1^s$.
\end{Prop}

\begin{Lem}
Falls $\abs{\cdot}_1$ und $\abs{\cdot}_2$ äquivalent sind, dann gilt:
\[ \abs{x}_1 < 1 \quad \Leftrightarrow \quad \abs{x}_2 < 1
\]
\end{Lem}

\begin{proof}
Seien $d_1$ und $d_2$ die zugehörigen Metriken.
\[ \abs{x}_1 < 1 
\quad \Leftrightarrow \quad  \abs{x}_1^n \xrightarrow{n \to \infty}
\quad \Leftrightarrow \quad x^n \xrightarrow{n \to \infty}_{d_1} 0
\quad \Leftrightarrow \quad x^n \xrightarrow{n \to \infty}_{d_2} 0
\quad \Leftrightarrow \quad \abs{x}_2 < 1 
\]
\end{proof}

\begin{Lem}
Sei $\abs{\cdot}_1$ eine nicht-triviale Metrik. Dann gilt:

Es gibt $s \in \R$ mit $\abs{\cdot}_2 = \abs{\cdot}_1^s$ genau dann, wenn
für alle $x,y \in K^\times$ gilt, dass
\[\log \abs{x}_1 \log \abs{y}_2 = \log \abs{x}_2 \log \abs{y}_1.
\]
In diesem Fall ist $s = \frac{\log \abs{x}_2}{\log \abs{x}_1}$, falls $x \in K^\times$ mit $\abs{x}_1 \neq 1$.
\end{Lem}


\begin{proof}
\enquote{$\Rightarrow$}
$\log \abs{x}_1 \log \abs{y}_2 = s \log \abs{x}_1 \log \abs{y}_1 = \log \abs{x}_2 \log \abs{y}_1$
	
\bigskip \enquote{$\Leftarrow$} 
Wähle $x\in K^\times$ mit $\abs{x}_1 \neq 1$ und setze $s = \frac{\log \abs{x}_2}{\log \abs{x}_1}$.
Dann gilt für alle $y \in K^\times$:
\begin{align*}
\log \abs{y} _2 = \frac{\log \abs{x}_2}{\log \abs{x}_1} \log \abs{y} _1
= s \log \abs{y}_1
\end{align*}
\end{proof}


\begin{proof}[Proof of Proposition 5.1.7]
\enquote{$\Leftarrow$}  Die Kreisscheiben um $x$ mit Radius $\varepsilon$ bzgl. $\abs{\cdot}_1$ sind genau die Kreisscheiben um $x$ mit Radius $\varepsilon^s$ bzgl. $\abs{\cdot}_2$.
Daher erhalten wir die selben Topologien.

\bigskip \enquote{$\Rightarrow$} 
Ohne Einschränkung sei $\abs{\cdot}_1$ nicht-trivial. Wähle $x\in K^\times$ mit $\abs{x} \neq 1$.
Sei nun $y \in K^\times$ und definiere $\alpha = \frac{\log \abs{y}_1}{\log \abs{x}_1}$ so, dass
$\abs{y}_1 = \abs{x}_1^\alpha$.

\bigskip
\textbf{(1)} Zeige: $\abs{y}_2 \leq \abs{x}_2^\alpha$

Wähle Folge $\frac{m_i}{n_i} \in \Q$ mit $\frac{m_i}{n_i} \xrightarrow{i} \alpha$ und $\frac{m_i}{n_i} \geq \alpha$ für alle $i$. Dann gilt:
\begin{align*}
\abs{y}_1
= \abs{x}_1^\alpha
\leq \abs{x}_1^{\frac{m_i}{n_i}}
&\Leftrightarrow \abs{y}_1^{n_i}
\leq \abs{x}_1^{m_i} \\
&\Leftrightarrow \abs{\frac{y^{n_i}}{x^{m_i}}}_1 \leq 1 \\
&\Leftrightarrow \abs{\frac{y^{n_i}}{x^{m_i}}}_2 \leq 1 \\
&\Leftrightarrow \abs{y}_2
\leq \abs{x}_2^{\frac{m_i}{n_i}}
\end{align*}
Mit $i \longrightarrow \infty$ erhalte $\abs{y_2} \leq \abs{x}_2^\alpha$.

\bigskip \textbf{(2)} Erhalte analog $\abs{y}_2 \geq \abs{x}_2^\alpha$ durch Folge $\frac{m_i}{n_i} \in \Q$ mit $\frac{m_i}{n_i} \xrightarrow{i} \alpha$ und $\frac{m_i}{n_i} \leq \alpha$ für alle $i$.

\bigskip Insgesamt: $\abs{y}_2 = \abs{x}_2^\alpha$

\bigskip Es gilt also
\[ \log \abs{y}_2 = \alpha \log \abs{x}_2
= \frac{\log \abs{y}_1}{\log \abs{x}_1} \log \abs{x}_2.
\]
Verwende Lemma 5.1.9 und erhalte für alle $y,y' \in K^\times$, dass
\[ \log \abs{y}_1 \log \abs{y'}_2
= \log \abs{y}_1 \frac{\log \abs{y'}_1}{\log \abs{x}_1} \log \abs{x}_2
= \log \abs{y}_2 \log \abs{y'}_1.
\]
Lemma 5.1.9 liefert nun $\abs{\cdot}_2 = \abs{\cdot}_1^s$ mit
$s = \frac{\log \abs{x}_2}{\log \abs{x}_1} >0$ nach Lemma 5.1.8.
\end{proof}

\begin{Bem}
Für beliebige $s>0$ gilt \textbf{nicht}: $\abs{\cdot}_1$ Norm $\Rightarrow$ $\abs{\cdot}_2^s$ Norm 
\end{Bem}

\begin{Bsp}
Die \textbf{Betrags-Norm} auf $\Q$ ist definiert durch
\[ \abs{\cdot} \colon \Q \to \R_{\geq 0}, \ q\mapsto \abs{q} = \begin{cases}
q, &\mbox{} q \geq 0, \\
-q, &\mbox{} q \leq 0.
\end{cases}
\]
\end{Bsp}

\begin{recall*}
$v_\p \colon \Q \to \Z \cup \{\infty \}$ p-adische Bewertung:
\begin{itemize}
\item Für $z \in \Z$ gilt: $v_p(z) = h \Leftrightarrow z =ap^h$ mit $\ggT(a,p)=1$
\item Für $q = \frac{x}{y} \in \Q^\times$ mit $x,y \in \Z$ gilt: $v_p(q) = v_p(x)-v_p(y)$
\item $v_p(0) = \infty$
\end{itemize}

\textbf{Es gilt:} $v_p$ ist \textbf{diskrete Bewertung}:
\begin{enumerate}[(i)]
\item $v_p(x) = \infty\Leftrightarrow x = 0$
\item $v_p(xy) = v_p(x) + v_p(y)$
\item $v_p(x+y) \geq \min \left\{ v_p(x), v_p(y) \right\}$
\end{enumerate}
\end{recall*}

\begin{Bem}
Definiere $\abs{\cdot}_p \colon Q \to \R_{\geq 0}$ durch $\abs{0}_p = 0$ und 
$\abs{x}_p = p^{-v_p(x)}$ für $x \neq 0$. Dann ist $\abs{\cdot}_p$ eine Norm, die \textbf{p-adische Norm}.
\end{Bem}


\begin{proof}
(i) und (ii) in Definition 5.1.1 folgen aus (i) und (ii) in Beispiel 5.1.12.
(iii) in Definition 5.1.1 folgt aus (iii) in Beispiel 5.1.12: Für $x,y \neq 0$ gilt:
\begin{align*}
\abs{x+y}_p
&= p^{-v_p(x+y)} \\
&\leq p^{- \min \left\{ v_p(x), v_p(y) \right\}} \\
&= p^{\max \left\{ -v_p(x), -v_p(y) \right\}} \\
&=\max\left\{ \abs{x}_p, \abs{y}_p \right\} \\
&\leq \abs{x}_p + \abs{y}_p \\
\end{align*}
\end{proof}

\begin{Bem}
\begin{enumerate}[(i)]
\item Es gilt sogar die \textbf{verschärfte Dreiecksungleichung}:
\[ \abs{x+y}_p \leq \max\left\{ \abs{x}_p, \abs{y}_p \right\}
\text{ für alle } x,y \in K
\]
\item Für $z \in \Z$ gilt 
\[ \abs{z}_p = \abs{ 1+ \cdots +1}_p \leq \abs{1}_p = 1.
\]
\item Die Norm $\abs{\cdot}$ auf $\Q$ definiert durch $\abs{x} = q^{-v_p(x)}$ mit beliebigem $q>1$ ist äquivalent zu $\abs{\cdot}_p$.
\end{enumerate}
\end{Bem}

\begin{Bsp}
\begin{enumerate}[(i)]
\item $\abs{1}_p = \abs{2}_p  = \abs{11}_p =1$
\item $\abs{5}_p = \frac{1}{5}$, $\abs{100}_p = \frac{1}{25}$
\item $\abs{\frac{1}{5}}_p = 5$, $\abs{\frac{1}{25}}_p = 25$
\end{enumerate}	
\end{Bsp}


