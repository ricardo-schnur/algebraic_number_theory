% !TeX spellcheck = de_DE
\rhead{30 Mai 2018}

\begin{proof}
\begin{enumerate}[(1)]
\item \OE \ $a_n=1$, denn bei Division durch $a_n$ wird das Newton-Polygon nur verschoben und die Nullstellen ändern sich nicht.
\item Seien nun $\alpha_1, \dots, \alpha_n$ die Nullstellen von $f$ in $L$ mit Vielfachheiten. Diese seien so sortiert, dass $w(\alpha_1) \leq w(\alpha_2) \leq \ldots \leq w(\alpha_n)$.\\
Genauso seien: $w(\alpha_1) = w(\alpha_2)=\ldots = w(\alpha_{s_1})=m_1 \quad$ ($s_1$-mal die Bewertung $m_1$).\\
$w(\alpha_{s_1+1}) = w(\alpha_{s_1+2})=\ldots = w(\alpha_{s_2})=m_2 $ usw.\\
$w(\alpha_{s_t+1}) = w(\alpha_{s_t+2})=\ldots = w(\alpha_{s_{t+1}})=m_{t+1} $ mit $s_{t+1}=n$, $(s_{t+1}-s_t)$-mal Bewertung $m_{t+1}$\\
mit $m_1 < m_2 < \dots < m_{t+1}$.
\item \underline{Erinnerung (Algebra}:
\[\prod_{c=1}^n (X-\alpha_i) = \sum_{d=0}^n (-1)^d s_d (\alpha_1, \dots, \alpha_n) X^{n-d}\]
mit $s_d(T_1, \dots, T_n)$ elementarsymmetrisches Polynom von Grad d
\[s_d:=\sum_{(i_1, \dots, i_d)} T_{i_1} \cdot \ldots \cdot T_{i_d}\]
das heißt:
\begin{align*}
a_{n-1}&=(-1)(\alpha_1+\dots+\alpha_n)\\
a_{n-2}&=\sum_{(i,j)} \alpha_i \alpha_j\\
a_{n-3}&=(-1)\sum_{(i,j,k)} \alpha_i \alpha_j\alpha_k\\
&\vdots\\
a_0&=(-1)^n \alpha_1 \cdot \ldots \cdot \alpha_n
\end{align*}
\item Sei nun $S=\{(i,v(a_i)) \ | \ i \in \{0, \dots, n\}\}$. Aus (3) folgt:
\begin{align*}
v(a_n)&=v(1)=0\\
v(a_{n-1})&=v(\alpha_1 +\dots + \alpha_n) \geq \min \{w(\alpha_1), \dots, w(\alpha_n)\}=m_1\\
v(a_{n-2})&=v(\sum_{(i,j)} \alpha_i \alpha_j) \geq \min \{w(\alpha_i \alpha_j) \ | \ (i,j) \}=2 m_1\\
&\dots\\
v(a_{n-s_1})&=v(\sum_{(i_1,\dots, i_{s_1})} \alpha_{i_1} \dots \alpha_{i_{s_1}}) \stackrel{(*)}{\geq}\min \{w(\alpha_1 \dots \alpha_{s_1})\}=s_1 m_1\\
\end{align*}
Bei $(*)$ gilt sogar Gleichheit, da für alle anderen $s_1$-Tupel $(i_1, \dots, i_{s_1})$ gilt \linebreak $w(\alpha_{i_1} \dots \alpha_{i_{s_1}})>w(\alpha_1 \dots \alpha_{s_1})$.\\
Sei $e_1=$ Strecke von $(n-s_1, v(a_{n-s_1}))$ bis $(n, v(a_n))$, dann hat $e_1$ die Steigung $-m_1$ und kombinatorische Länge $s_1$.\\
Die Ecken von $c_1$ liegen in $S$. Alle Punkte $(i, v(a_i))$ mit $n>i>n-s_1$ liegen auf der Strecke $e_1$ oder darüber. Genauso
\begin{align*}
v(a_{n-s_1-1})&\geq s_1m_1+m_2\\
v(a_{n-s_1-2})&\geq s_1m_1+2m_2\\
&\dots\\
v(a_{n-s_2})&=s_1m_1+(s_1-s_2)m_2
\end{align*}
Also $e_2=$ Strecke von $(n-s_2, v(a_{n-s_2}))$ nach $(n-s_1, v(a_{n-s_1}))$ hat die Steigung $-m_2$ und kombinatorische Länge $s_2-s_1$. Die Ecken von $e_2$ liegen in $S$ und alle Punkte $(i, v(a_i))$ mit $n-s_2 < i<n-s_1$ liegen auf oder oberhalb von $e_2$. Auf die gleiche Weise erhalte Kanten $e_3, e_4, \dots, e_{t+1}$, jeweils gilt:
\begin{itemize}
\item $e_i$ ist die Strecke von $(n-s_i, s_1m_1+\dots+(s_i-s_{i-1})m_i)$ nach $(n-s_{i-1}, s_1m_1+\dots+(s_{i-1}-s_{i-2})m_{i-1})$
\item $e_i$ hat die Steigung $-m_i$ und hat kombinatorische Länge $s_i-s_{i-1}$
\item Die Ecken von $e_i$ liegen in $S$
\item Alle Punkte $j, v(a_j))$ in $S$ mit $n-s_i \leq j \leq n-{s_{i-1}}$ liegen auf oder oberhalb von $e_i$.
\end{itemize}
Schließlich ist wegen $m_1 < m_2 < \dots < m_{t+1}$ der Bereich, der vom Kantenzug $(e_1, \dots, e_{t+1})$ nach oben berandet wird konvex. Damit ist $(e_1, \dots , e_{t+1})$ das Newton-Polygon.
\end{enumerate}
\end{proof}

\begin{Bem}
Sei $(K,v)$ bewerteter Körper und $L|K$ eine Körpererweiterung, auf die sich $v$ in eindeutiger Weise zu Bewertung $w$ fortsetzt.
\begin{enumerate}[i)]
\item Dann gilt für $\sigma \in \Aut_K(L): w \circ \sigma = w$.\\
Es folgt insbesondere für den Bewertungsring $\O_w$ zu $w$ und das maximale Ideal $\p_w: \sigma(\O_w)=\O_w$ und $\sigma(\p_w)=\p_w$. Somit induziert $\sigma$ einen Automorphismus $\overline{\sigma}: \kappa_w \to \kappa_w$ auf dem Restklassenkörper $\kappa_w=\O_w/\p_w \Rightarrow$ Wir erhalten Homomorphismus $\Aut_K(L) \to \Gal(\kappa_w|\kappa_v)$ mit $\kappa_v$ Restklassenkörper zu $v$.
\item Falls $L=Z(f)$ ist mit $f=a_nX^n+\dots+a_0 \in \O_v[X]$ für ein irreduzibles Polynom $f$ über dem Bewertungsring $\O_v$ zu $v$, dann liegen die Nullstellen $\alpha$ von $f$ in $\O_w$.
\end{enumerate}
\end{Bem}

\begin{proof}
i) folgt direkt, da Fortsetzung $w$ eindeutig und damit $w \circ \sigma = w$.\\
ii) Sei $\mu_\alpha$ die Multiplizität von $\alpha$ in $f \Rightarrow f(X)= \prod_{\sigma \in \Aut_K(L)} (X-\sigma(\alpha))^{\mu_\alpha}$.\\
Annahme: $\alpha \not \in \O_w \Rightarrow w(\alpha)<0 \stackrel{i)}{\Rightarrow} w(\sigma(\alpha))<0 \ \forall \sigma \in \Aut_K(L)$\\
$\Rightarrow$ Für $a_0=(-1)^{\deg(f)} \prod_{\sigma \in \Aut_K(L)} \sigma(\alpha)^{\mu_\alpha}$ gilt: $w(a_0)<0 \Rightarrow a_0 \not \in \O_v$ \Lightning.
\end{proof}

\begin{Prop}
In der Situation von Prop. 8.7 schreibe
\[ f(X)=a_n \cdot \prod_{j=1}^{t+1} f_j(x) \text{ mit } f_j(X)= \prod_{w(\alpha_i)=m_j} (X-\alpha_i)\]
Dann gilt: $f_j \in K[X]$.
\end{Prop}

\begin{proof}
Wiederum \OE \ $a_n=1$
\begin{enumerate}[(1)]
\item Falls $f$ irreduzibel ist, dann sind alle Nullstellen $\alpha_1, \dots, \alpha_n$ von $f$ konjugiert durch $\Gal(L|K) \stackrel{Bem. 8.8}{\Longrightarrow} w(\alpha_1)=\dots=w(\alpha_n) \Rightarrow t+1=1 \Rightarrow f_j \in K[X]$.
\item Sei $f$ nun beliebig. Wir machen Induktion nach $\deg(f)$:\\
\glqq $n=1$ \grqq: \checkmark\\
\glqq $n \to n+1$ \grqq: Sei $p(X) \in K[X]$ das Minimalpolynom von $\alpha_1$. Alle Nullstellen von $p$ haben die gleiche Bewertung $m_1 \Rightarrow p$ ist Teiler von $f_1$. Sei $g_1(X):= f_1(X)/p(X)$ und $g(X):=f(X)/p(X) \in K[X]$\\
$\Rightarrow \deg(g) < \deg(f)$ und $g(X)=\underbrace{g_1(X)}_{m_1} \cdot \underbrace{f_2(X)}_{m_2} \cdot \ldots \cdot \underbrace{f_{t+1}(X)}_{m_{t+1}}$\\
$\stackrel{I.V.}{\Rightarrow} g_1, f_2, \dots f_{t+1} \in K[X]$.
\end{enumerate}
\end{proof}

\begin{Kor}
In der Situation von Prop. 8.7 gilt:\\
Falls $f(X)=a_n X^n + a_{n-1}X^{n-1} + \dots + a_0$ irreduzibel, dann $|f| \stackrel{\text{Def.}}{=} \max\{|a_n|, |a_{n-1}|, \dots, |a_0|\}=\max\{|a_n|, |a_0|\}$.
\end{Kor}

\begin{proof}
Das Newton-Polygon besteht nur aus einer Kante mit den Eckpunkten $(0, v(a_0))$ und $(n, v(a_n))$.
\end{proof}

\begin{Prop}
Ein Körper mit Bewertung $v$ ist Henselsch $\iff v$ lässt sich auf jede algebraische Körpererweiterung $L$ eindeutig fortsetzen.
\end{Prop}
