\rhead{11 April 2018}

\section{Archimedische und nicht-archimedische Normen}

\textbf{Ab jetzt:} Betrachte nur nicht-triviale Normen!

\begin{defi}
Eine Norm $\abs{\cdot} \colon K \to \Rnn$ heißt \textbf{nicht-archimedisch} falls ein $C \in \Rnn$ existiert mit $\abs{n} \leq C$ für alle $n\in \N$.
Andernfalls heißt  $\abs{\cdot}$ \textbf{archimedisch}.

Hierbei: $n = 1+ \dots + 1 \in K$
\end{defi}

\begin{Prop}
Eine Norm $\abs{\cdot} \colon K \to \Rnn$ ist nicht-archimedisch genau dann, wenn die verschärfte Dreiecksungleichung gilt, d.h. für alle $x,y \in K$:
\[ \abs{x+y} \leq \max \left\{ \abs{x}, \abs{y} \right\}
\] 
\end{Prop}

\begin{proof}
\enquote{$\Leftarrow$} $\abs{n} \leq \max\{ \abs{1}, \dots, \abs{1} \} = \abs{1} = 1$

\bigskip \enquote{$\Rightarrow$}
Sei $C\in\Rnn$ mit $\abs{n} \leq C$ für alle $n\in\N$. Für $x,y \in K$ können wir ohne Einschränkung annehmen, dass $\max \{\abs{x}, \abs{y}  \} = \abs{x}$ gilt und somit
\[ \abs{x}^j \abs{y}^{n-j} \leq \abs{x}^n
\]
für alle $j \in\N_0$ und daher
\begin{align*}
\abs{x+y}^n
&= \abs{(x+y)^n}
= \abs{ \sum_{j=0}^{n} \binom{n}{j} x^j y^{n-j}  }\\
&\leq \sum_{j=0}^{n} \binom{n}{j} \abs{x}^j \abs{y}^{n-j} 
\leq \sum_{j=0}^{n} C \abs{x}^n
=(n+1)C\abs{x}^n
\end{align*}
Es folgt $ \abs{x+y} \leq \sqrt[n]{n+1} \sqrt[n]{C} \abs{x}$ so, dass
$ \abs{x+y} \leq  \abs{x} = \max \{\abs{x}, \abs{y}  \} $.
\end{proof}


\begin{Bem}
Ist $\abs{\cdot}$ nicht-archimedisch, so gilt:
\[ \abs{x} \neq \abs{y} \quad \Rightarrow \quad 
\abs{x+y} = \max \left\{ \abs{x}, \abs{y} \right\} 
\]
\end{Bem}

\begin{proof}
Ohne Einschränkung sei $\abs{y} < \abs{x}$.

\enquote{$\leq$} Gilt nach der verschärften Dreiecksungleichung.

\enquote{$\geq$} $\abs{y} < \abs{x} = \abs{x+y-y} \leq \max \{ \abs{x+y}, \abs{y} \}
= \abs{x+y}$
\end{proof}


\begin{defi}
Wir betrachten eine Abbildung $v\colon K \to \R \cup \{\infty \}$.
\begin{enumerate}[(i)]
\item $v$ heißt \textbf{Bewertung}, falls für alle $x,y \in K$ die folgenden Bedingungen erfüllt sind:
		\begin{enumerate}[(i)]
			\item $v(x) = \infty\Leftrightarrow x = 0$
			\item $v(xy) = v(x) + v(y)$
			\item $v(x+y) \geq \min \left\{ v(x), v(y) \right\}$
		\end{enumerate}
\item $v$ heißt \textbf{diskrete Bewertung}, falls $v$ eine Bewertung ist und $v(K) \cap \R_{>0}$ ein Minimum $s$ besitzt. Dies ist genau dann der Fall, wenn $v(K) = s\Z \cup \{\infty \}$.
\item Zwei Bewertungen $v_1, v_2 \colon K \to \R_{>0}$ heißen \textbf{äquivalent}, falls ein $r \in \R\bs\{0\}$ existiert mit $v_1(x) = rv_2(x)$ für alle $x \in K^\times$.

Die Bewertung $v_{\triv} \colon K \to \R \cup \{\infty \}$ mit $v_{\triv}(x) = 0$ für alle $x \in K^\times$
heißt \textbf{triviale Bewertung}.
\end{enumerate}
\end{defi}


\begin{Prop}
Die Bewertungen auf $K$ entsprechen bijektiv den nicht-archimedischen Normen, wobei die triviale Bewertung der trivialen Norm entspricht und äquivalente Bewertungen äquivalenten Normen entsprechen.
\end{Prop}

\begin{proof}
Seien $q>1$,
\[ M_1 = \left\{ v\colon K \to \R_\cup\{\infty \} \, | \, v \text{ ist Bewertung}   \right\}
\]
und 
\[ M_2 = \left\{ \abs{\cdot} \colon K \to \R \, | \, \abs{\cdot} \text{ ist Norm}   \right\}.
\]
Definiere weiterhin
\[ f = f_q \colon M_1 \to M_2, \colon v \mapsto \abs{\cdot}
\text{ mit } \abs{x} = q^{-v(x)} \text{ für } x \in K^\times
\]
sowie
\[ g = g_q \colon M_2 \to M_1, \colon \abs{\cdot} \mapsto v
\text{ mit } v(x)=-\log_q \abs{x} \text{ für } x \in K^\times.
\]
Es gilt: $f,g$ sind invers zueinander und bilden trivial (äuivalent) auf trivial (äuivalent) ab.
\end{proof}


\begin{Bem}
Im Beweis von Proposition 5.2.5 spielt das gewählte $q>1$ bis auf Äquivalenz keine Rolle.
\end{Bem}

\setcounter{Satz}{0}
\begin{Satz}[Satz von Ostrowski]
Jede nicht-triviale Norm $\norm{\cdot}$ auf $\Q$ ist äquivalent zu $\abs{\cdot}_\infty$ oder zu $\abs{\cdot}_p$.
\end{Satz}

\begin{proof}
\textbf{Fall 1:} $\norm{\cdot}$ ist archimedisch

Schreibe $\abs{\cdot} = \abs{\cdot}_\infty$. Da $\norm{\cdot}$ archimedisch ist, gibt es $n_0 \in \N_{>1}$ mit $\norm{n_0}>1$.

\bigskip Zeige: Für alle $m,n\in \N$ gilt:
$\log \norm{m} \log \abs{n} = \log \abs{m} \log \norm{n}$

Dann folgt die Behauptung aus Lemma 5.1.9.

\bigskip Für $n=1$ oder $m=1$ ist obige Gleichung offensichtlich erfüllt.
Schreibe $m=a_rn^r + \dots + a_0$ mit $a_i \in \{ 0, \dots, n-1 \}$ und $a_r \neq 0$.
Insbesondere gelten: 

(A) $0 \leq r \leq \frac{\log m}{\log n}$ 
\qquad 
(B) $\norm{a_i} = \underbrace{\norm{1+ \dots + 1}}_{\text{$a_i$-mal}} \leq n$

\bigskip \textbf{Fall A:} $\norm{n} \geq 1$:
\begin{align*}
\norm{m}
&= \norm{a_0 + \dots + a_rn^r}
\leq \sum_{i=0}^{r} \norm{a_i} \norm{n}^i\\
&\overset{\text{(B)}}{ \leq} \sum_{i=0}^{r}  n \norm{n}^r
=(r+1)n\norm{n}^r \\
&\leq \left(\, \frac{\log m}{\log n} +1  \right)n \norm{n}^{ \frac{\log m}{\log n} }
\end{align*}
Es folgt für alle $k \in \N$, dass
\[ \norm{m^k} \leq \left(\, \frac{k\log m}{\log n} +1  \right) n \left(\, \norm{n}^{ \frac{\log m}{\log n} } \right)^k
\]
und somit
\[ \norm{m} \leq \left(\, \frac{k\log m}{\log n} +1  \right)^{\frac{1}{k}} n^{\frac{1}{k}}  \norm{n}^{ \frac{\log m}{\log n} } ,
\]
also $\norm{m} \leq \norm{n}^{ \frac{\log m}{\log n} }$.

\bigskip \textbf{Fall B:} $\norm{n} < 1$:

Analog wie oben erhalte $\norm{m} \leq \left(\, \frac{\log m}{\log n} +1  \right) n$
und damit auch $\norm{m^k} \leq \left(\, \frac{k\log m}{\log n} +1  \right) n$ für alle $k \in \N$.
Es folgt
\[ \norm{m} \leq \left(\, \frac{k\log m}{\log n} +1  \right)^{\frac{1}{k}} n^{\frac{1}{k}}
\]
so, dass der Grenzübergang $k \to \infty$ schließlich $\norm{m} \leq 1$ liefert.

\bigskip Insgesamt: $\norm{m} \leq \max \left\{  \norm{n}^{ \frac{\log m}{\log n} }, 1  \right\}$

Wähle jetzt $m =n_0$, d.h. $\norm{m} = \norm{n_0}>$ so, dass obige Abschätzung
$\norm{m} \leq \norm{n}^{ \frac{\log m}{\log n} }$ liefert. Damit gilt insbesondere $\norm{n}>1$ für alle $n \in \N_{>1}$.

Also tritt Fall B nicht auf und für $n,m \in \N_{>1}$ gilt: $\norm{m} \leq \norm{n}^{ \frac{\log m}{\log n} }$ 
Durch Vertauschen der Rollen von $m$ und $n$ erhalten wir ebenso 
$\norm{n} \leq \norm{m}^{ \frac{\log n}{\log m} }$ und damit
\[ \norm{m} \leq \norm{n}^{ \frac{\log m}{\log n} } \leq \norm{m}
\]
so, dass
\[ \norm{m} =\norm{n}^{ \frac{\log m}{\log n} }.
\]
Es folgt
\[ \log\norm{m} =\frac{\log\abs{m}}{\log\abs{n}} \log\norm{n}
\]
und Lemma 5.1.9 liefert schließlich $\norm{\cdot} = \abs{\cdot}^s$ mit
\[ s = \frac{\log\norm{n_o}}{\log\abs{n_0}} > 0.
\]
Dies zeigt die Äquivalenz von $\norm{\cdot}$ und $\abs{\cdot}$.

\bigskip \textbf{Fall 2:} $\norm{\cdot}$ ist nicht-archimedisch

AZT1 [IV, Prop. 1.3]: Die einzigen diskreten Bewertungen auf $\Q$ sind die $p$-adischen Bewertungen (bis auf Skalierung). Im Beweis wird \textbf{nicht} verwendet, dass $v$ diskret ist.
Also folgt die Behauptung aus Proposition 5.2.5
\end{proof}


\begin{Bem}
Sei $v\colon K \to \R \cup \{\infty \} $ eine Bewertung und
$\O = \{ x \in K \, | \, v(x) \geq 0 \}$. Dann gilt für alle $x \in K^\times$, dass $x \in \O$ oder $\frac{1}{x} \in \O$.
\end{Bem}


\begin{defi}
Sei $R$ ein Integritätsbereich und $K = \Quot(R)$. Dann nennen wir $R$ \textbf{Bewertungsring}, falls
für alle $x \in K$ gilt, dass $x \in R$ oder $\frac{1}{x} \in R$.
\end{defi}

\begin{Prop}
Sei $R$ ein Bewertungsring. Dann gilt:
\begin{enumerate}[(i)]
\item $R^\times = \left\{ x \in R \, | \, \frac{1}{x} \in R \right\}$
\item $\m = \left\{ x \in R \, | \, \frac{1}{x} \not\in R \right\}$ ist das einzige maximale Ideal von $R$, insbesondere ist $R$ lokal.
\item $R$ ist ganz abgeschlossen.
\end{enumerate}
\end{Prop}

