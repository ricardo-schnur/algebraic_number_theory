\rhead{02 Juli 2018}

\section{Fortsetzungen von Bewertungen}

\underline{Ziel:} $(K,v)$ Körper mit Bewertung (nicht notwendigerweise Henselsch), $L|K$ algebraische Körpererweiterung.\\
\underline{Frage:} Gibt es eine Fortsetzung $w$ von $v$ nach $L$ und wenn ja, wie viele?\\
\underline{Idee:} Gehe über zu Vervollständigung:
\begin{tikzcd}
L \arrow[r, hook] & \overline{K} \arrow[r, hook] & (\overline{K_v}, \overline{v})\\
(K,v) \arrow[u, hook, "alg."] \arrow[rr, hook]&& (K_v, \hat{v}) \arrow[u, hook, "alg."]
\end{tikzcd}

\begin{Not}
Ab jetzt sei
\begin{itemize}
\item $(K,v)$ ein fest gewählter Körper.
\item $(K_v, \hat{v})$ die Vervollständigung von $(K,v)$
\item $\overline{K_v}$ der algebraische Abschluss von $K_v$ und
\item $\overline{v}$ die eindeutige Fortsetzung von $\hat{v}$ auf $\overline{K_v}$
\end{itemize}
\end{Not}

\begin{Prop}
Sei $L|K$ eine algebraische Körpererweiterung $\Rightarrow v$ hat mindestens eine Fortsetzung $w$ auf $L$.
\end{Prop}

\begin{proof}
Wähle Einbettung $\tau: L \hookrightarrow \overline{K_v}$. Diese existiert, da $L|K$ algebraisch ist. Dann ist:
\[ w_\tau := \overline{v} \circ \tau \quad \text{ eine Bewertung auf } L\]
\end{proof}

\begin{Bem}
In der Situation von Prop. 10.2 gilt:\\
Ist $\sigma \in \Gal(\overline{K_v}|K_v)$ und $\tau':=\sigma \circ \tau$, dann ist $w_{\tau'}=w_\tau$.
\end{Bem}

\begin{proof}
Eindeutigkeit im Fortsetzungssatz von Theorem 5 $\Rightarrow \overline{v} \circ \sigma = \overline{v}$
$\Rightarrow w_{\tau'} = \overline{v} \circ \sigma \circ \tau = \overline{v} \circ \tau = w_\tau$.
\end{proof}

\begin{Prop}
Sei $L|K$ endliche Körpererweiterung und $\tau: L \hookrightarrow \overline{K_v}$ eine Einbettung. Sei $(\hat{L}, \hat{w}_\tau)$ die Vervollständigung von $(L, w_\tau)$ mit $w_\tau = \overline{v} \circ \tau$.

Die Einbettung $\tau : L \hookrightarrow \overline{K_v}$ lässt sich fortsetzen zu $\hat{\tau}: \hat{L} \hookrightarrow \overline{K_v}$. Insbesondere gilt: $\hat{L}|K_v$ ist algebraisch.
\end{Prop} 

\begin{proof}
\begin{tikzcd}
&\hat{L} \arrow[d, dotted, "\hat{\tau}"] &\\
(L, w_\tau) \arrow[r, hook, swap, "\tau"]  \arrow [ru, hook]& (\tau(L) \cdot K_v, \overline{v}) \arrow[draw=none]{r}[auto=false, sloped]{\subseteq}&(\overline{K_v}, \overline{v})\\
\arrow[r, Rightarrow] & \ & \\
(K,V) \arrow[uu, hook, "endl."] \arrow[r, hook] & (K_v, \hat{v}) \arrow[uu, hook,swap, "endl."]& 
\end{tikzcd}


Sei $E:= \tau(L) \cdot K_v$ das Kompositum von $L$ und $K_v$ in $\overline{K_v}$.\\
$L|K$ endlich $\Rightarrow E|K_v$ endlich $\stackrel{Thm\ 5}{\Rightarrow} \hat{v}$ hat eindeutige Fortsetzung auf $E$ und ist bezüglich dieser vollständig. Die Bewertung $\overline{v}|_E$ ist eine solche. Außerdem gilt: $w_\tau = \overline{v} \circ \tau = \overline{v}|_E \circ \tau.$\\
$\Rightarrow \tau : (L, w_\tau) \to (E, \overline{v}|_E)$ erhält die Norm und bildet in einen vollständigen Körper ab $\Rightarrow \tau $ lässt sich fortsetzen zu $\hat{\tau}: \hat{L} \hookrightarrow E$ und zwar eindeutig.
\end{proof}

\begin{Prop}
$L|K$ endliche Körpererweiterung, $w$ eine beliebige Fortsetzung von $v$. Dann gilt für die Vervollständigung $(\hat{L}, \hat{w})$:\\
$\hat{L}|K_v$ ist endlich.
\end{Prop}

\begin{proof}
\begin{tikzcd}
L \arrow[r, hook] & L \cdot K_v \subseteq \hat{L}\\
K \arrow[u, hook] \arrow[r, hook] &K_v \arrow[u, hook]
\end{tikzcd}
Sei $L \cdot K_v$ das Kompositum von $L$ und $K_v$ in $\hat{L} \Rightarrow L \cdot K_v | K_v$ ist endlich.\\
Thm. 5 $\Rightarrow L \cdot K_v$ ist vollständig bezüglich $\hat{w}|_{L\cdot K_v} \Rightarrow L \cdot K_v = \hat{L}$.
\end{proof}

\begin{Bem}
Für $L|K$ algebraisch stimmt die analoge Aussage in Prop. 10.5 nicht, d.h.: $L|K$ algebraisch $\not \Rightarrow \hat{L} | K_v$ algebraisch.
\end{Bem}

\begin{Bsp*}[ohne Beweis]
$K=\Q_p$ und $L=\overline{\Q_p}$, aber $\hat{\overline{\Q_p}} | \Q_p$ ist nicht algebraisch ($\hat{\overline{\Q_p}} \neq \overline{\Q_p})$.
\begin{tikzcd}
\overline{\Q_p} \arrow[r, hook] & \hat{\overline{\Q_p}}\\
\Q_p \arrow[u, hook, "alg."] \arrow[r, hook, "id"]& \Q_p \arrow[u, hook, "\underline{nicht} \ alg."]
\end{tikzcd}
\end{Bsp*}

\begin{defi}
Sei $L|K$ algebraisch, $w$ Fortsetzung von $v$ auf $L$.\\
$I:=\{(L', w') \ | \ L' \text{ ist endlicher Zwischenkörper von } L|K, w'= w|_{L'} \}$\\
Notiere für $(L', w') \in I$ mit $(\hat{L}',\hat{w}')$ die Vervollständigung von $(L', w')$.\\
Dann heißt $L_w:= \bigcup_{(L',w') \in I} \hat{L}'$ \textbf{\emph{Lokalisierung von $L$ bezüglich $w$}} und ist mit der Bewertung $\hat{w}|_{L_w}$ versehen, die wir ebenfalls mit $\hat{w}$ bezeichnen.\\
\end{defi}

Beachte: $L_1' \subseteq L_2'$ induziert Einbettung $\hat{L}_1' \hookrightarrow \hat{L}_2'$. Mit Standardtrick erhalte, dass $L_w$ ein Körper ist. Insbesondere: $L|K$ endlich $\Rightarrow L_w$ ist Vervollständigung von $(L,w)$.

\begin{Prop}
$L|K$ algebraisch $\Rightarrow L_w | K_v$ algebraisch.
\end{Prop}

\begin{proof}
folgt aus Prop 10.5 und Def. 10.7
\end{proof}

\begin{Prop}
Betrachte Körperkette $K \subseteq L \subseteq \overline{K_v}$.\\
Sei $L$ versehen mit der Bewertung $w:=\overline{v}|_L$. Dann gilt:
\begin{enumerate}[i)]
\item $L_w \cong L \cdot K_v$
\item $L K_v$ ist der Abschluss von $L$ in $\overline{K_v}$
\end{enumerate}
\end{Prop}

\begin{proof}
\begin{enumerate}[i)]
\item Für $L|K$ endlich wie in Prop. 10.5. Für $L|K$ unendlich:
\begin{enumerate}[(1)]
\item Erhalte Morphismus $\sigma: L_w \hookrightarrow L \cdot K_v$ durch \glqq Zusammensetzen\grqq \ der Morphismen $\sigma': L'_{w'} \hookrightarrow L' \cdot K_v$ für die endlichen Körpererweiterungen $L'|K$.
\item $\Bild(\sigma)$ enthält $K_v$ und $L \Rightarrow \Bild(\sigma) \supseteq K_v \cdot L \Rightarrow \sigma$ ist Isomorphismus zwischen $L_w$ und $K_v \cdot L$.
\end{enumerate}
\item folgt aus i), da $L$ dicht in $L_w$.
\end{enumerate}
\end{proof}

\begin{Satz}
Sei $L|K$ algebraische Körpererweiterung. Dann gilt:
\begin{enumerate}[i)]
\item $v$ hat eine Fortsetzung $w$ auf $L$, die aber nicht eindeutig sein muss.
\item Für jede Fortsetzung $w$ gilt: $w=w_\tau:=\overline{v} \circ \tau$ für eine Einbettung $\tau: L \hookrightarrow \overline{K}_v$.
\item $w_\tau = w_{\tau'} \iff \tau$ und $\tau'$ sind konjugiert über $K_v$, d.h.: $\exists \ \sigma \in \Gal(\overline{K}_v | K_v)$ mit $\tau'=\sigma \circ \tau$.
\end{enumerate}
\end{Satz}

\begin{proof}
\begin{enumerate}[i)]
\item Prop. 10.2
\item Sei $(L_w, \hat{w})$ die Lokalisierung von $(L,w)$. Wähle $K_v$-Einbettung $\hat{\tau}: L_w \hookrightarrow \overline{K}_v$\\
$\Rightarrow$ Erhalte folgendes Diagramm:
\[\begin{tikzcd}
(L,w) \arrow[r, hook] & (L_w, \hat{w}) \arrow[r, hook, "\hat{\tau}"] & (\overline{K}_v, \overline{v})\\
(K,v) \arrow[u, hook, "alg."] \arrow[r, hook] &(K_v, \hat{v}) \arrow[u, hook, "alg."] \arrow[ru, hook]
\end{tikzcd}\]
Thm. 5 $\Rightarrow \hat{w}= \overline{v} \circ \hat{\tau}$ Sei $\tau := \hat{\tau}|_L \Rightarrow w = \overline{v} \circ \tau \stackrel{Def.}{=} w_\tau$.
\item \glqq $\Leftarrow$\grqq : s. Bem 10.3\\
\glqq $\Rightarrow$\grqq : Sei $w_\tau = w_{\tau'}$. Betrachte das folgende Diagramm
\[
\begin{tikzcd}
(\overline{K}_v, \overline{v}) && (\overline{K}_v, \overline{v}) \arrow[ll,swap, "\overline{\sigma}"]\\
L_2 K_v \arrow[draw=none]{u}[auto=false, sloped]{\subseteq} && L_1 K_v \arrow[draw=none]{u}[auto=false, sloped]{\subseteq} \arrow[ll,swap, "\hat{\sigma}"]\\
L_2:=\tau'(L) \arrow[draw=none]{u}[auto=false, sloped]{\subseteq} & L \arrow[l, "\tau'"]\arrow[r, "\tau"] & L_1:=\tau(L) \arrow[draw=none]{u}[auto=false, sloped]{\subseteq} \arrow[ll, bend right=15, swap, "\sigma=\tau' \circ \tau^{-1}"]\\
&(K,v) \arrow[u, hook]&
\end{tikzcd}
\]
Sei $L_1 := \tau(L)$ und $L_2:=\tau'(L)$ Prop 10.9 $\Rightarrow L_1$ liegt dicht in $L_1 \cdot K_V \cong (L_1)_{\overline{v}|_{L_1}}$ und $L_2 \cdot K_V \cong (L_2)_{\overline{v}|_{L_2}}$.\\
$\overline{v}  \circ \tau = \overline{v} \circ \tau' \Rightarrow \sigma$ erhält die Bewertung, i.e. $\overline{v}(\sigma(\alpha)) = \overline{v}(\alpha)$.\\
$\Rightarrow \sigma$ hat Fortsetzung $\hat{\sigma}: L_1 K_v \to L_2 K_v$, denn:\\
$x \in L_1 K_v \cong (L_1)_{\overline{v}|_{L_1}} \Rightarrow \exists
 $ endl. Zwischenkörper $L'$ von $L_1 |K$ mit $x \in L'_{w'}$, wobei $w'=\overline{v}|_{L'}$.\\
 Wähle Folge $x_n$ in $L'$ mit $x_n \to x$. Insbesondere: $x_n$ ist Cauchy-Folge. Da $\sigma$ die Bewertung erhält, ist auch $\sigma(x_n)$ Cauchy-Folge. Da $\sigma(x_n)$ in $\sigma(L')$ und $\sigma(L')$ ist endlicher Zwischenkörper von $L_2 | K$, folgt: $\sigma(x_n)$ hat Limes $x'$ in $L_2 K_v$. Definiere $\hat{\sigma}(x):=x'$.\\
 Beachte: $\hat{\sigma}$ ist $K_v$-Homomorphismus\\
 $\Rightarrow \hat{\sigma}$ setzt sich fort zu $\overline{\sigma}: \overline{K_v} \to \overline{K_v}$ mit $\overline{\sigma} \in \Gal(\overline{K_v} | K_v)$. Nach Konstruktion gilt: $\overline{\sigma} \circ \tau = \sigma \circ \tau = \tau'$.
\end{enumerate}
\end{proof}