\rhead{23 April 2018}


\begin{Bsp}
\begin{enumerate}[(i)]
\item Für das inverse System in Beispiel 5.4.2 (i) erhalten wir als projektiven Limes
\[ \hat{\Z} = \left\{ (a_n)_{n\in\N} \, | \, a_n \in \Z/n\Z, n \text{ teilt } m \Rightarrow a_m \equiv a_n \mod n  \right\}
\]
mit komponentenweiser Addition.
\item Für Beispiel 5.4.2 (ii) erhalten wir
\begin{align*}
\hat{G} = \{ &(g_N)_{\text{$N$ Normalteiler von $G$ mit endlichem Index}} \, | \, g_N \in G/N, \\
&M \subset N \Rightarrow g_N \mod N = g_M \mod M  \}.
\end{align*}
$\hat{G}$ heißt \textbf{proendliche Komplettierung} von $G$.
\end{enumerate}
\end{Bsp}

\begin{Bem}
Analog erhält man projektive Limiten unter anderem in folgenden Kategorien:
\begin{center}
\Mengen, \Ringe, \Rmod, \kalg
\end{center}
\end{Bem}

\begin{defi}
Eine \textbf{topologische Gruppe} besteht aus
\begin{itemize}
\item[(1)] einer Gruppe $G$ und
\item[(2)] einer Topologie $T$ auf $G$ so, dass
\item[(A)] die Gruppenabbildung $G\times G \to G, (a,b) \mapsto ab$ stetig ist und
\item[(B)] die Inversenabbildung $G\to G, g \mapsto g^{-1}$  stetig ist.
\end{itemize}
Ein  \textbf{Morphismus} $\varphi$ zwischen topologischen Gruppen ist ein stetiger Gruppenhomomorphismus.
\end{defi}

\begin{Bem}
Topologische Gruppen bilden mit Morphismen eine Kategorie, \TopGroups.
\end{Bem}

\begin{defi}
Sei $I$ eine Indexmenge und $(X_i,T_i)_{i\in I}$ eine Familie topologischer Räume. Sei $X=\prod_{i\in I}X_i$ das kartesische Produkt und $p_i \colon X \to X_i, \, (x_k) \mapsto x_i$.
Dann ist die \textbf{Produkttopologie} $T$ die kleinste Topologie auf $X$ für die alle $p_i$ stetig sind. Das heißt
\[ \B = \left\{ p_i^{-1}(U) \,|\, i \in I, U \subset X_i \text{ offen}   \right\}
\]
bildet eine Subbasis von $T$.
\end{defi}

\begin{remin}
\begin{itemize}
\item $\B \subset T$ heißt \textbf{Subbasis}, falls jede offene Menge $U\in T$ eine beliebige Vereinigung von Mengen, die endliche Schnitte von Mengen in $\B$ sind, ist.
\item Für $x\in X$ heißt $\B_x\subset T$ \textbf{Umgebungsbasis} von $x$, falls:
\begin{enumerate}
\item[(1)] Für alle $U \in \B_x$ gilt $x \in U$.
\item[(2)] Für jedes $V\in T$ mit $x \in V$ existiert ein $U\in B_x$ mit $U \subset V$.
\end{enumerate}
\end{itemize}
\end{remin}

\begin{Bem}
Sei $G$ eine topologische Gruppe mit neutralem Element $1$.
\begin{enumerate}
\item[(1)] Jedes $g\in G$ definiert einen Homöomorphismus (\textbf{kein} Automorphismus)
\[ \Phi_g \colon G \to G, \, x\mapsto gx
\]
mit Umkehrabbildung $\Phi_{g^{-1}}$.
\item[(2)] $\B_{(1)}$ ist Umgebungsbasis von $1$ genau dann, wenn
\[ g \B_{(1)} = \{ gU \, | \, U \in \B_{(1)} \}
\]
eine Umgebungsbasis von $g$ ist für jedes $g \in G$.
\item[(3)] Ist $G$ zusammenhängend so gilt
\[ G = \bigcup_{n \in \N} \B_{(1)}^n.
\]
\end{enumerate}
\end{Bem}


\begin{Bem}
In der Kategorie $\TopSpaces$ und in der Kategorie $\TopGroups$ existieren jeweils projektive Limiten.
\end{Bem}


\begin{proof}
Analog zu Proposition 5.4.5 betrachte auf $\prod_{i\in I}A_i$ die Produkttopologie und auf $A$ die Spurtopologie davon.
\end{proof}

\begin{Bem}
Die proendliche Komplettierung $\hat{G}$ einer Gruppe $G$ wird zu einer topologischen Gruppe wie folgt:
\begin{itemize}
\item[(1)] Versehe die Quotienten $G/N$ mit der diskreten Topologie $\Pot(G/N)$.
\item[(2)] Bilde den projektiven Limes in der Kategorie der topologischen Gruppen $\TopGroups$.
\end{itemize}
\end{Bem}

\begin{defi}
Eine topologische Gruppe $G$ heißt \textbf{proendlich}, falls $G$ isomorph zum projektiven Limes von endlichen Gruppen, jeweils versehen mit der diskreten Topologie, ist.
\end{defi}

\begin{Bem}
Analog definiert man \textbf{topologische Ringe} als Ringe $(R,+,\cdot)$ mit $(R,+)$ topologische Gruppe und $\cdot \colon R\times R\to R$ stetig und sieht, dass auch für diese projektive Limiten existieren.
\end{Bem}



\section{Ganze $p$-adische Zahlen}

\textbf{Ziel:} Zeige, dass
\[ \Z_p \cong \limproj_{n\in \N} \Z /p^n \Z,
\]
beziehungsweise allgemeiner, dass
\[ \hat{\O} \cong \limproj_{n\in\N} \O / \p^n.
\]


\textbf{Sei ab jetzt:}
\begin{itemize}
\item $K$ ein Körper, $v$ eine diskrete Bewertung auf $K$ und $\abs{\cdot} = q^{-v(\cdot)}$ die zugehörige Norm für ein $q>1$
\item $\hat{K}$ die Vervollständigung von $K$ bezüglich $\abs{\cdot}$, $\hat{v}$ die zugehörige Bewertung und  $\abs{\cdot} = q^{-\hat{v}(\cdot)}$ die zugehörige Norm
\item $\O$ der Bewertungsring zu $v$ mit maximalem Ideal $\p= \Pi \cdot\O$ mit $v(\Pi)=1$
\item $\hat{\O}$ der Bewertungsring zu $\hat{v}$ mit maximalem Ideal $\hat{\p}= \Pi \cdot\hat{\O}$
\end{itemize}


\begin{Bem}
Da $v$ diskrete Bewertung ist gilt $v(K)=\hat{v}(\hat{K})$, denn für $c_n \xrightarrow{n\to\infty}x$ gilt $\hat{v}(x) = \lim\limits_{n\to \infty} v(c_n)$.
\end{Bem}

Ohne Einschränkung sei ab jetzt $v(k) = \hat{v}(\hat{K}) = \Z \cup \{\infty \}$.


\begin{Prop}
Für alle $n\in\N_0$ gilt $\hat{\O} / \hat{\p}^n \cong \O / \p^n$.
\end{Prop}

\begin{proof}
Für $x \in \O$ gilt:
\[ x \in \p^n \Leftrightarrow v(x) \geq n \Leftrightarrow \hat{v}(x) \geq n \Leftrightarrow x \in \hat{\p}^n
\]
Also steigt $i \colon O \hookrightarrow \hat{\O}$ ab zu einem injektiven Morphismus
$\O/ \p^n \hookrightarrow \hat{\O} / \hat{\p}^n$. Dieser ist surjektiv, denn zu $x \in \hat{\O}$ existiert eine Folge $(c_k)$ in $K$ mit $c_k \to x$. Da $v$ eine diskrete Bewertung ist, gilt ohne Einschränkung $c_k \in \O$, also ist für $k$ groß genug $\abs{x-c_k}<q^{-n}$ und damit $v(x-c_n)>n$.
Also $c_k \equiv x \mod \hat{\p}^n$. Folglich wird $c_k \mod \p^n$ abgebildet auf $x \mod \hat{\p}^n$.
\end{proof}

\begin{Prop}
Es gilt $ \hat{\O} \cong \limproj_{n\in\N} \O / \p^n$ in der Kategorie $\TopRings$ von topologischen Ringen. Hierbei erhält $\hat{\O}$ die von der Norm $\abs{\cdot}$ induzierte Topologie und die Ringe $\O / \p^n$ die diskrete Topologie.
\end{Prop}








